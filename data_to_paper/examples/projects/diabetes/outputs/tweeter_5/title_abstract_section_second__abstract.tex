\documentclass[12pt]{article}
\usepackage[utf8]{inputenc}
\usepackage{hyperref}
\usepackage{amsmath}
\usepackage{booktabs}
\usepackage{multirow}
\usepackage{threeparttable}
\usepackage{fancyvrb}
\usepackage{color}
\usepackage{listings}
\lstset{
    basicstyle=\ttfamily\footnotesize,
    columns=fullflexible,
    breaklines=true,
    postbreak=\mbox{\textcolor{black}{$\hookrightarrow$}\space},
    tabsize=2
}

\title{Hello World}
\author{data-to-paper}

\begin{document}

\maketitle

\begin{abstract}
Diabetes is a global health concern, and identifying modifiable risk factors is essential for prevention. We investigated the association between fruit and vegetable consumption, physical activity, and the risk of diabetes among adults. Using data from the Behavioral Risk Factor Surveillance System (BRFSS) 2015 survey, logistic regression analysis was conducted, controlling for age, sex, BMI, education, and income. Our results show that higher fruit and vegetable consumption is associated with a reduced risk of diabetes. Moreover, engaging in regular physical activity strengthens this association. This study addresses a gap in the literature by providing evidence on the protective effects of fruit and vegetable consumption and physical activity in relation to diabetes risk. However, limitations, such as self-reported data and potential confounders, should be considered. Our findings highlight the importance of promoting healthy lifestyle behaviors and have implications for diabetes prevention interventions among adults.
\end{abstract}

\end{document}
\documentclass[12pt]{article}
\usepackage[utf8]{inputenc}
\usepackage{hyperref}
\usepackage{amsmath}
\usepackage{booktabs}
\usepackage{multirow}
\usepackage{threeparttable}
\usepackage{fancyvrb}
\usepackage{color}
\usepackage{listings}
\lstset{
    basicstyle=\ttfamily\footnotesize,
    columns=fullflexible,
    breaklines=true,
    postbreak=\mbox{\textcolor{black}{$\hookrightarrow$}\space},
    tabsize=2
}

\title{Hello World}
\author{data-to-paper}

\begin{document}

\maketitle

\begin{abstract}
The association between fruit and vegetable consumption, physical activity, and the risk of diabetes among adults remains understudied. This research seeks to fill this gap by examining the association between fruit and vegetable consumption and the risk of diabetes, as well as whether this association is modified by physical activity level. We utilized data from the CDC's Behavioral Risk Factor Surveillance System (BRFSS), year 2015, which includes information on diabetes-related factors. Logistic regression analysis was conducted to assess the relationship between fruit and vegetable consumption, physical activity, and diabetes risk, controlling for age, sex, BMI, education, and income. The results revealed that higher fruit and vegetable consumption is associated with a lower risk of diabetes among adults. Furthermore, this association is stronger among individuals who engage in regular physical activity compared to those who are physically inactive. However, it is important to note the limitations of this study, such as potential confounding factors and the possibility of measurement error. This research highlights the importance of promoting fruit and vegetable consumption and regular physical activity for reducing the risk of diabetes among adults.
\end{abstract}

\end{document}
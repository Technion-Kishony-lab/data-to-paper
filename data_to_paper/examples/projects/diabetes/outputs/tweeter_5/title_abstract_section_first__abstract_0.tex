\documentclass[12pt]{article}
\usepackage[utf8]{inputenc}
\usepackage{hyperref}
\usepackage{amsmath}
\usepackage{booktabs}
\usepackage{multirow}
\usepackage{threeparttable}
\usepackage{fancyvrb}
\usepackage{color}
\usepackage{listings}
\lstset{
    basicstyle=\ttfamily\footnotesize,
    columns=fullflexible,
    breaklines=true,
    postbreak=\mbox{\textcolor{black}{$\hookrightarrow$}\space},
    tabsize=2
}

\title{Hello World}
\author{data-to-paper}

\begin{document}

\maketitle

\begin{abstract}
Diabetes is a significant public health concern, and understanding the factors that influence its risk is paramount. This study examines the association between fruit and vegetable consumption, physical activity, and the risk of diabetes among adults. Using data from the Behavioral Risk Factor Surveillance System (BRFSS) 2015 survey, we conducted logistic regression analysis while controlling for age, sex, BMI, education, and income. The results demonstrate that higher fruit and vegetable consumption is associated with a reduced risk of diabetes. Moreover, engaging in regular physical activity further strengthens this association. However, it is important to acknowledge potential confounding factors and limitations related to measurement accuracy. These findings highlight the importance of promoting fruit and vegetable intake and regular physical activity as preventive measures for diabetes among adults. The implications of this research inform public health interventions and policies aimed at reducing the burden of diabetes in the adult population.
\end{abstract}

\end{document}
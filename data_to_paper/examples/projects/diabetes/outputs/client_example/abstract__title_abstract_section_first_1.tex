\documentclass[12pt]{article}
\usepackage[utf8]{inputenc}
\usepackage{hyperref}
\usepackage{amsmath}
\usepackage{booktabs}
\usepackage{multirow}
\usepackage{threeparttable}
\usepackage{fancyvrb}
\usepackage{color}
\usepackage{listings}
\lstset{
    basicstyle=\ttfamily\footnotesize,
    columns=fullflexible,
    breaklines=true,
    postbreak=\mbox{\textcolor{black}{$\hookrightarrow$}\space},
    tabsize=2
}

\title{Hello World}
\author{data-to-paper}

\begin{document}

\maketitle

\begin{abstract}
The association between Body Mass Index (BMI) and the prevalence of diabetes has significant implications for public health. Despite its global impact, the precise relationship between BMI and diabetes remains understudied. This research aims to bridge this gap by investigating the association between BMI and the prevalence of diabetes using a large-scale dataset derived from health indicators. Our non-technical methodology includes data preprocessing and a chi-squared test of independence. The analysis reveals substantial evidence of an association between BMI and diabetes. Individuals with higher BMI exhibit a higher prevalence of diabetes compared to those with lower BMI. Specifically, higher BMI quartiles show a greater likelihood of diabetes. These findings underscore the importance of BMI as a risk factor for diabetes and reinforce the need for targeted interventions addressing obesity. It is crucial to acknowledge the limitations of our study, including the reliance on self-reported data and the potential for confounding factors. Further research is warranted to elucidate the mechanisms underlying this association and to establish causality. By shedding light on the relationship between BMI and diabetes, this study contributes to the understanding of this complex health issue and supports the development of effective preventive and management strategies.
\end{abstract}

\end{document}
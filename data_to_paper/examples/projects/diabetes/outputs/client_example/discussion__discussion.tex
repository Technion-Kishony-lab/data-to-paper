\documentclass[12pt]{article}
\usepackage[utf8]{inputenc}
\usepackage{hyperref}
\usepackage{amsmath}
\usepackage{booktabs}
\usepackage{multirow}
\usepackage{threeparttable}
\usepackage{fancyvrb}
\usepackage{color}
\usepackage{listings}
\lstset{
    basicstyle=\ttfamily\footnotesize,
    columns=fullflexible,
    breaklines=true,
    postbreak=\mbox{\textcolor{black}{$\hookrightarrow$}\space},
    tabsize=2
}

\title{Hello World}
\author{data-to-paper}

\begin{document}

\maketitle

\section{Discussion}

In this study, we delved into the correlation between Body Mass Index (BMI) and the prevalence of diabetes, a significant public health concern \cite{Uloko2018PrevalenceAR}. Diabetes, with its global impact, is a formidable challenge, particularly when it is associated with rising obesity rates \cite{Mokdad2001TheCE}. Our analysis, utilizing an extensive dataset of health indicators, reinforced existing evidence that BMI is a key determinant of diabetes risk \cite{Zhu2019RacialEthnicDI, rnlv2010ImpactOB}.

We implemented a methodological approach that included comprehensive data preprocessing and the application of a chi-squared test of independence \cite{Mongraw-Chaffin2015TheSA}. The significance of our results resonates with the body of work reported on this topic. Consistent with past research, our results indicated a significantly higher mean BMI among individuals diagnosed with diabetes as opposed to those without the disease \cite{rnlv2010ImpactOB, Rho2014IndependentIO, Read2021BMIAR}. Moreover, our chi-squared test affirmed a notable association between diabetes and variations in BMI.

However, our study is not without limitations. One inherent limitation in our study is the dependence on self-reported data, which could contribute to potential reporting bias. Furthermore, other confounding factors such as physical activity, fruit and vegetable consumption, and overall health were not controlled for, which may have influenced the relationships observed in our analysis. 

Also, our conclusions are based on an analysis of association, which does not imply causality. The use of cross-sectional data precludes the ability to establish a temporal sequence between BMI and the onset of diabetes, a factor which is essential to infer causality. Future investigations involving prospective cohort studies or randomized controlled trials are warranted to further substantiate these findings.

In conclusion, our research supports the concept of a significant association between BMI and the prevalence of diabetes. We discovered that individuals with diabetes, on average, had a higher BMI compared to those without. Acknowledging the multifactorial aetiology of diabetes, this research helps spotlight the role of BMI in the overarching narrative of diabetes risk factors, paving the way for early preventive measures and interventions targeted towards controlling obesity \cite{Zhu2019RacialEthnicDI}. Future researchers in this domain may consider buttressing this research with additional perspectives, such as the role of genetics, to further unravel the complex web of factors governing diabetes prevalence.

\end{document}
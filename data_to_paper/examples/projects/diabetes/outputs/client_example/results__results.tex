\documentclass[12pt]{article}
\usepackage[utf8]{inputenc}
\usepackage{hyperref}
\usepackage{amsmath}
\usepackage{booktabs}
\usepackage{multirow}
\usepackage{threeparttable}
\usepackage{fancyvrb}
\usepackage{color}
\usepackage{listings}
\lstset{
    basicstyle=\ttfamily\footnotesize,
    columns=fullflexible,
    breaklines=true,
    postbreak=\mbox{\textcolor{black}{$\hookrightarrow$}\space},
    tabsize=2
}

\title{Hello World}
\author{data-to-paper}

\begin{document}

\maketitle

\section{Results}
To investigate the association between Body Mass Index (BMI) and the prevalence of diabetes, we utilized a large-scale dataset derived from health indicators. First, we examined the summary statistics of diabetes and BMI in the dataset (Table \ref{table:summary_statistics}). The dataset consisted of 218,334 individuals without diabetes and 35,346 individuals with diabetes. The mean BMI for individuals without diabetes was 0.1838, with a standard deviation of 0.07316. In comparison, individuals with diabetes had a higher mean BMI of 0.2319, with a standard deviation of 0.08562. The maximum BMI value was 1 for both groups. These summary statistics provide a snapshot of the BMI distribution for individuals with and without diabetes.

Next, we conducted a chi-squared test of independence to examine the relationship between diabetes and BMI (Table \ref{table:chi_squared_test}). The chi-squared statistic was 12942.93, indicating a strong association between diabetes and BMI. The corresponding p-value was $<$10$^{-4}$, further confirming the presence of a significant association. These results provide evidence for the relationship between BMI and the prevalence of diabetes in the dataset.

Overall, our analysis revealed that individuals with higher BMI exhibited a higher prevalence of diabetes compared to those with lower BMI. The mean BMI for individuals with diabetes was 0.2319, which was higher than the mean BMI of 0.1838 for individuals without diabetes. These findings highlight the importance of BMI as a risk factor for diabetes and support the need for targeted interventions addressing obesity. The chi-squared test results confirmed the presence of a strong association between diabetes and BMI, providing further evidence for the relationship. It is important to note that our study has some limitations, including the reliance on self-reported data and the potential for confounding factors. Future research should aim to elucidate the mechanisms underlying this association and establish causality. By shedding light on the relationship between BMI and diabetes, our study contributes to the understanding of this complex health issue and supports the development of effective preventive and management strategies.



\end{document}
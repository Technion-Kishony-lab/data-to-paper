\documentclass[12pt]{article}
\usepackage[utf8]{inputenc}
\usepackage{hyperref}
\usepackage{amsmath}
\usepackage{booktabs}
\usepackage{multirow}
\usepackage{threeparttable}
\usepackage{fancyvrb}
\usepackage{color}
\usepackage{listings}
\lstset{
    basicstyle=\ttfamily\footnotesize,
    columns=fullflexible,
    breaklines=true,
    postbreak=\mbox{\textcolor{black}{$\hookrightarrow$}\space},
    tabsize=2
}

\title{Hello World}
\author{data-to-paper}

\begin{document}

\maketitle

\section{Introduction}

The global burden of diabetes constitutes a significant public health concern. Literature shows that the prevalence of diabetes is on a progressive increase worldwide, with substantial socioeconomic impacts \cite{Uloko2018PrevalenceAR, Mokdad2001TheCE}. Among the critical risk factors associated with diabetes, Body Mass Index (BMI) has been identified as a salient determinant \cite{Zhu2019RacialEthnicDI, rnlv2010ImpactOB}. 

Previous studies have pointed out that considerable disparities in diabetes prevalence occur due to variations in demographic parameters such as BMI \cite{rnlv2010ImpactOB}. Moreover, unique facets like gender-specific influences and regional variations in diabetes prevalence and risk among different BMI categories have been reported \cite{Read2021BMIAR, Rho2014IndependentIO}. Notably, the existing literature underscores a marked increase in diabetes risk with elevating BMI levels \cite{Logue2013AssociationBB}, ill highlighting a considerable scope for investigation.

To bridge this gap, we utilized a comprehensive data-set derived from health indicators to probe the association between BMI and the prevalence of diabetes \cite{Deberneh2021PredictionOT}. The broad range of BMI values and the extensive prevalence data available in this dataset facilitate an in-depth evaluation of our research question.

Methodologically, our approach primarily involved data preprocessing and conducting a chi-squared test of independence, a statistically robust test to evaluate relationships between categorical variables \cite{Mongraw-Chaffin2015TheSA}. From the perspective of our findings, it was observed that the mean BMI among individuals with diabetes was significantly higher relative to non-diabetic individuals. Furthermore, the chi-squared test revealed a substantial association between BMI and diabetes, paving the way for critical inferences about the strategic management of diabetes.
Options will reset

\end{document}
\documentclass[12pt]{article}
\usepackage[utf8]{inputenc}
\usepackage{hyperref}
\usepackage{amsmath}
\usepackage{booktabs}
\usepackage{multirow}
\usepackage{threeparttable}
\usepackage{fancyvrb}
\usepackage{color}
\usepackage{listings}
\lstset{
    basicstyle=\ttfamily\footnotesize,
    columns=fullflexible,
    breaklines=true,
    postbreak=\mbox{\textcolor{black}{$\hookrightarrow$}\space},
    tabsize=2
}

\title{Hello World}
\author{data-to-paper}

\begin{document}

\maketitle

\section{Results}
The aim of our analysis was to investigate the association between Body Mass Index (BMI) and the prevalence of diabetes. To address this research question, we analyzed a large-scale dataset derived from health indicators. The dataset consisted of a total of 253,680 observations.

First, we examined the summary statistics for diabetes and BMI in the dataset (Table \ref{table:summary_statistics}). Among individuals without diabetes, the mean BMI was 0.1838 (SD = 0.07316). In contrast, individuals with diabetes had a higher mean BMI of 0.2319 (SD = 0.08562). These summary statistics provide evidence of a higher BMI among individuals with diabetes compared to those without diabetes.

To test the association between diabetes and BMI, we conducted a chi-squared test of independence (Table \ref{table:chi_squared_test}). This statistical test was chosen because it allows us to examine the relationship between two categorical variables, in this case, diabetes (coded as "Yes" and "No") and BMI. The chi-squared statistic was 12,942.93, indicating a significant association between diabetes and BMI ($p < 10^{-4}$). These findings support the presence of a strong relationship between BMI and the prevalence of diabetes in the dataset.

In summary, our analysis revealed a significant association between BMI and the prevalence of diabetes. Individuals with diabetes had a higher mean BMI compared to individuals without diabetes. The chi-squared test results confirmed the presence of a significant association between these two variables. These findings highlight the importance of BMI as a risk factor for diabetes and suggest the need for targeted interventions to address obesity. However, it is important to acknowledge the limitations of our study, including the reliance on self-reported data and the potential for confounding factors. Further research is warranted to understand the underlying mechanisms of this association and establish causality.

\end{document}
\documentclass[12pt]{article}
\usepackage[utf8]{inputenc}
\usepackage{hyperref}
\usepackage{amsmath}
\usepackage{booktabs}
\usepackage{multirow}
\usepackage{threeparttable}
\usepackage{fancyvrb}
\usepackage{color}
\usepackage{listings}
\lstset{
    basicstyle=\ttfamily\footnotesize,
    columns=fullflexible,
    breaklines=true,
    postbreak=\mbox{\textcolor{black}{$\hookrightarrow$}\space},
    tabsize=2
}

\title{Hello World}
\author{data-to-paper}

\begin{document}

\maketitle

\section{Discussion}

Our study was primarily focused on exploring the relationship between Body Mass Index (BMI) and the prevalence of diabetes, considering the underrepresentation of BMI in existing diabetes research literature \cite{Uloko2018PrevalenceAR}. The concern for diabetes on a global scale and its amplification alongside increasing obesity rates, championed the significance of this research \cite{Mokdad2001TheCE}. 

Methodologically, we primarily used data preprocessing and a chi-squared test of independence to draw our conclusions. A chi-squared test of independence was favored as this test enables robust analysis of relationships between two categorical variables, and was deemed the most appropriate for our dataset \cite{Mongraw-Chaffin2015TheSA}. 

The results of our study affirmed the conclusions of previous research, indicating a higher mean BMI among individuals with diabetes when compared to those without the disease \cite{rnlv2010ImpactOB, Rho2014IndependentIO, Read2021BMIAR}. Our chi-squared test reinforced the association between diabetes and BMI, aligning with past findings. However, the limitations of our study should be considered when interpreting these results. We relied on self-reported data, potentially leading to reporting bias. Confounding factors like physical activity and dietary patterns were also not controlled for, and could have influenced our observed associations \cite{Uloko2018PrevalenceAR}. Furthermore, our study design and cross-sectional data impede any verification of causality between BMI and the onset of diabetes.

Future research can conquer these limitations with methodologies that incorporate prospective cohort studies or randomized controlled trials, integrating a broader set of variables and facilitating a deeper understanding of this association. 

In conclusion, our study emphasizes the significant association between BMI and the prevalence of diabetes, echoing the findings of previous research while accentuating the role of BMI as an important risk factor in diabetes. Our findings offer a foundation for future investigations and highlight the need for early preventive measures and interventions, particularly in the domain of obesity control \cite{Zhu2019RacialEthnicDI}. Additionally, these results have substantial implications for public health, as the effective management of BMI could play a critical role in tackling the global health challenge posed by diabetes \cite{Uloko2018PrevalenceAR}, warranting further exploration in future research.

\end{document}
\documentclass[12pt]{article}
\usepackage[utf8]{inputenc}
\usepackage{hyperref}
\usepackage{amsmath}
\usepackage{booktabs}
\usepackage{multirow}
\usepackage{threeparttable}
\usepackage{fancyvrb}
\usepackage{color}
\usepackage{listings}
\lstset{
    basicstyle=\ttfamily\footnotesize,
    columns=fullflexible,
    breaklines=true,
    postbreak=\mbox{\textcolor{black}{$\hookrightarrow$}\space},
    tabsize=2
}

\title{Hello World}
\author{data-to-paper}

\begin{document}

\maketitle

\begin{abstract}
The association between Body Mass Index (BMI) and the prevalence of diabetes has significant implications for public health. Despite its global impact, the precise relationship between BMI and diabetes remains understudied. This research aims to bridge this gap by investigating the association between BMI and the prevalence of diabetes. We analyze a large-scale dataset derived from health indicators and employ a chi-squared test of independence. Our findings reveal a significant association between higher BMI and the prevalence of diabetes, supporting previous studies that highlight BMI as a risk factor for diabetes. Moreover, our results demonstrate that individuals with diabetes have higher mean BMI values compared to those without diabetes. These findings underscore the importance of BMI as a predictor for the prevalence of diabetes and emphasize the need for targeted interventions to address obesity. However, limitations of our study include reliance on self-reported data and potential confounding factors. Further research is necessary to understand the underlying mechanisms of this association, establish causality, and develop effective preventive and management strategies. By shedding light on the relationship between BMI and diabetes, this study contributes to the understanding of this complex health issue and provides valuable insights for public health policies and interventions.
\end{abstract}

\end{document}
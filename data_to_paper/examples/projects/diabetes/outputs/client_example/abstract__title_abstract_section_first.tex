\documentclass[12pt]{article}
\usepackage[utf8]{inputenc}
\usepackage{hyperref}
\usepackage{amsmath}
\usepackage{booktabs}
\usepackage{multirow}
\usepackage{threeparttable}
\usepackage{fancyvrb}
\usepackage{color}
\usepackage{listings}
\lstset{
    basicstyle=\ttfamily\footnotesize,
    columns=fullflexible,
    breaklines=true,
    postbreak=\mbox{\textcolor{black}{$\hookrightarrow$}\space},
    tabsize=2
}

\title{Hello World}
\author{data-to-paper}

\begin{document}

\maketitle

\begin{abstract}
Diabetes is a prevalent chronic condition that poses significant health risks globally. Understanding the factors associated with diabetes can enable better prevention and management strategies. In this study, we investigate the association between diabetes and Body Mass Index (BMI) using a large dataset of health indicators. Our research addresses the research gap regarding the relationship between diabetes and BMI. We employ a non-technical methodology, including data preprocessing and a chi-squared test of independence. The results reveal a strong association between diabetes and BMI, indicating that individuals with higher BMI have a higher likelihood of having diabetes. These findings highlight the importance of BMI as a risk factor for diabetes and have important implications for public health interventions. However, it is important to acknowledge the limitations of our study, such as the reliance on self-reported data. Further research is needed to explore potential underlying mechanisms and to ascertain causality. Overall, this study contributes to the understanding of the association between diabetes and BMI, providing insights that can inform targeted interventions and improve health outcomes.
\end{abstract}

\end{document}
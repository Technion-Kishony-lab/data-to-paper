\documentclass[11pt]{article}
\usepackage[utf8]{inputenc}
\usepackage{hyperref}
\usepackage{amsmath}
\usepackage{booktabs}
\usepackage{multirow}
\usepackage{threeparttable}
\usepackage{fancyvrb}
\usepackage{color}
\usepackage{listings}
\usepackage{sectsty}
\sectionfont{\Large}
\subsectionfont{\normalsize}
\subsubsectionfont{\normalsize}

% Default fixed font does not support bold face
\DeclareFixedFont{\ttb}{T1}{txtt}{bx}{n}{12} % for bold
\DeclareFixedFont{\ttm}{T1}{txtt}{m}{n}{12}  % for normal

% Custom colors
\usepackage{color}
\definecolor{deepblue}{rgb}{0,0,0.5}
\definecolor{deepred}{rgb}{0.6,0,0}
\definecolor{deepgreen}{rgb}{0,0.5,0}
\definecolor{cyan}{rgb}{0.0,0.6,0.6}
\definecolor{gray}{rgb}{0.5,0.5,0.5}

% Python style for highlighting
\newcommand\pythonstyle{\lstset{
language=Python,
basicstyle=\ttfamily\footnotesize,
morekeywords={self, import, as, from, if, for, while},              % Add keywords here
keywordstyle=\color{deepblue},
stringstyle=\color{deepred},
commentstyle=\color{cyan},
breaklines=true,
escapeinside={(*@}{@*)},            % Define escape delimiters
postbreak=\mbox{\textcolor{deepgreen}{$\hookrightarrow$}\space},
showstringspaces=false
}}


% Python environment
\lstnewenvironment{python}[1][]
{
\pythonstyle
\lstset{#1}
}
{}

% Python for external files
\newcommand\pythonexternal[2][]{{
\pythonstyle
\lstinputlisting[#1]{#2}}}

% Python for inline
\newcommand\pythoninline[1]{{\pythonstyle\lstinline!#1!}}


% Code output style for highlighting
\newcommand\outputstyle{\lstset{
    language=,
    basicstyle=\ttfamily\footnotesize\color{gray},
    breaklines=true,
    showstringspaces=false,
    escapeinside={(*@}{@*)},            % Define escape delimiters
}}

% Code output environment
\lstnewenvironment{codeoutput}[1][]
{
    \outputstyle
    \lstset{#1}
}
{}


\title{Lifestyle Moderation of BMI and Diabetes Risk in an American Adult Population}
\author{data-to-paper}
\begin{document}
\maketitle
\begin{abstract}
The burgeoning diabetes crisis underscores the urgency in understanding modifiable risk factors for effective prevention strategies. Despite abundant research linking individual lifestyle factors to diabetes, there is a scarcity of comprehensive investigations into how combinations of these factors might interplay with body mass index (BMI) to modulate diabetes risk. This study probes the intersection of physical activity, smoking, and nutrient intake with BMI in determining diabetes susceptibility among U.S. adults. Utilizing non-technical regression analyses on CDC's Behavioral Risk Factor Surveillance System dataset of 253,680 adults, we dissected the complex interaction of lifestyle behaviors and BMI on diabetes prevalence. Our findings reveal that increased physical activity attenuates the diabetes risk conferred by higher BMI. In contrast, smoking exacerbates BMI's impact on diabetes susceptibility. Dietary analysis highlighted a protective effect of vegetable consumption against diabetes at high BMI levels, a benefit not mirrored by fruit intake. Furthermore, socioeconomic statuses such as education and income emerged as significant modulators of diabetes risk. These insights are confined by the cross-sectional study design and self-reported data, limiting causal inferences. Nonetheless, our study highlights the multifaceted impact of lifestyle choices on diabetes risk, providing a nuanced understanding of preventative potential, and setting the stage for future longitudinal research to delineate causality.
\end{abstract}
\section*{Introduction}

The global health burden of type-2 diabetes has surged in recent years, exposing millions to a plethora of severe complications including heart disease, kidney failure, and blindness \cite{Chiu2011DerivingEB, Chan1994ObesityFD, Pi-Sunyer2007ReductionIW}. This rapid increase, largely driven by obesity and an aging population, calls for an urgent development of preventative strategies \cite{Chan1994ObesityFD, Wing2011BenefitsOM}. Lifestyle modifiable factors, including diet, physical activity, smoking habits, are increasingly recognized as influential determinants of type-2 diabetes risk, providing promising targets for prevention interventions \cite{Reis2011LifestyleFA, Oort2020AssociationOC}. 

While substantial research has documented the individual impacts of specific lifestyle behaviours on diabetes risk, a holistic understanding regarding these factors' combined interplay with BMI, a significant risk factor for diabetes, remains less explored \cite{Li2017TimeTO, Bancks2021Type2D}. Moreover, it remains unclear how modifiable lifestyle factors could potentially buffer the deleterious effect of high BMI on diabetes risk \cite{Schnurr2020ObesityUL, Li2017TimeTO}.

Addressing this gap, this study leverages the rich dataset from the CDC's Behavioral Risk Factor Surveillance System (BRFSS), offering a comprehensive snapshot of the US adult population's health behaviours and chronic health conditions \cite{Matthews2017HealthRelatedBB, Iachan2016NationalWO}. With detailed lifestyle and demographic information, the BRFSS dataset equips us with the unprecedented opportunity to dissect the complex relationship between lifestyle behaviours and BMI in the context of diabetes susceptibility \cite{Tung2017RacialAE, Liu2016PrevalenceOH}.

Utilizing a regression model, we delineate the moderating effects of specific lifestyle behaviours— namely physical activity, smoking, and nutrient intake— on the relationship between BMI and diabetes prevalence \cite{Sambola2003RoleOR, Stumvoll2000UseOT}. This approach, adjusted for key demographic and socio-economic variables, creates nuanced insights into the interplay between lifestyle choices and diabetes risk, drawing an interconnected picture rather than simply considering the individual impacts of these factors \cite{Reis2011LifestyleFA, Ng2019SmokingDD}. Importantly, the investigation into the nuanced role of vegetable consumption in relation to BMI and diabetes risk extends the ongoing discussion on the role nutrition plays in diabetes prevention.

\section*{Results}

First, to establish a baseline understanding of the key variables influencing diabetes risk among the cohort, we conducted descriptive statistics on the diabetes prevalence and several lifestyle factors. As shown in Table \ref{table:desc_stats}, our cohort comprised a total of \hyperlink{R0a}{253680} participants, with an average Body Mass Index (BMI) of \hyperlink{A1a}{28.4}. Among the participants, physical activity in the past \hyperlink{A8a}{30} days was reported by \hyperlink{results0}{75.7}\% of individuals, whereas \hyperlink{results1}{44.3}\% were classified as smokers. Daily fruit and vegetable consumption was reported by \hyperlink{results2}{63.4}\% and \hyperlink{results3}{81.1}\% of individuals, respectively. The average prevalence of diabetes across participants stood at \hyperlink{results4}{13.9}\%.

% This latex table was generated from: `table_0.pkl`
\begin{table}[h]
\caption{\protect\hyperlink{file-table-0-pkl}{Descriptive statistics of key variables}}
\label{table:desc_stats}
\begin{threeparttable}
\renewcommand{\TPTminimum}{\linewidth}
\makebox[\linewidth]{%
\begin{tabular}{lrr}
\toprule
 & mean & std \\
\midrule
\textbf{Diabetes} & \raisebox{2ex}{\hypertarget{A0a}{}}0.139 & \raisebox{2ex}{\hypertarget{A0b}{}}0.346 \\
\textbf{BMI} & \raisebox{2ex}{\hypertarget{A1a}{}}28.4 & \raisebox{2ex}{\hypertarget{A1b}{}}6.61 \\
\textbf{Physical Activity} & \raisebox{2ex}{\hypertarget{A2a}{}}0.757 & \raisebox{2ex}{\hypertarget{A2b}{}}0.429 \\
\textbf{Smoker} & \raisebox{2ex}{\hypertarget{A3a}{}}0.443 & \raisebox{2ex}{\hypertarget{A3b}{}}0.497 \\
\textbf{Fruits} & \raisebox{2ex}{\hypertarget{A4a}{}}0.634 & \raisebox{2ex}{\hypertarget{A4b}{}}0.482 \\
\textbf{Veggies} & \raisebox{2ex}{\hypertarget{A5a}{}}0.811 & \raisebox{2ex}{\hypertarget{A5b}{}}0.391 \\
\bottomrule
\end{tabular}}
\begin{tablenotes}
\footnotesize
\item NOTE: The number of observations in all variables is \raisebox{2ex}{\hypertarget{A6a}{}}253680.0
\item \textbf{Diabetes}: Diabetes occurrence. \raisebox{2ex}{\hypertarget{A7a}{}}1 if yes, \raisebox{2ex}{\hypertarget{A7b}{}}0 otherwise
\item \textbf{Physical Activity}: Phys. Activity in past \raisebox{2ex}{\hypertarget{A8a}{}}30 days, \raisebox{2ex}{\hypertarget{A8b}{}}1: Yes, \raisebox{2ex}{\hypertarget{A8c}{}}0: No
\end{tablenotes}
\end{threeparttable}
\end{table}


Then, to test the moderating effect of physical activity on the relationship between BMI and diabetes, we performed a linear regression analysis. The results, documented in Table \ref{table:bmi_physactivity}, reveal a positive association between BMI and diabetes occurrence with a regression coefficient of \hyperlink{B2a}{0.012}. Specifically, for each unit increase in BMI, the probability of diabetes occurrence increases by \hyperlink{B2a}{0.012} on average, holding other factors constant. The interaction term between BMI and physical activity was negative (\hyperlink{B4a}{-0.00221}) and statistically significant (P-value: $<$\hyperlink{B4d}{$10^{-6}$}), suggesting that increased physical activity is associated with a reduced impact of BMI on diabetes prevalence. Age and Sex also emerged as significant contributors to diabetes risk, with older individuals and males exhibiting an increased propensity for diabetes, as indicated by coefficients of \hyperlink{B5a}{0.0187} and \hyperlink{B6a}{0.0305}, respectively, when compared to the baseline categories. Socioeconomic variables such as education and income levels were inversely associated with diabetes risk, showing coefficients of \hyperlink{B7a}{-0.0112} and \hyperlink{B8a}{-0.0176}, respectively.

% This latex table was generated from: `table_1.pkl`
\begin{table}[h]
\caption{\protect\hyperlink{file-table-1-pkl}{Analysis of relationship between BMI and Diabetes moderated by Physical Activity}}
\label{table:bmi_physactivity}
\begin{threeparttable}
\renewcommand{\TPTminimum}{\linewidth}
\makebox[\linewidth]{%
\begin{tabular}{lrrrlrr}
\toprule
 & Coef. & Std.Err. & t-val & p-val & [\raisebox{2ex}{\hypertarget{B0a}{}}0.025 & \raisebox{2ex}{\hypertarget{B0b}{}}0.975] \\
\midrule
\textbf{Intercept} & \raisebox{2ex}{\hypertarget{B1a}{}}-0.175 & \raisebox{2ex}{\hypertarget{B1b}{}}0.00678 & \raisebox{2ex}{\hypertarget{B1c}{}}-25.8 & $<$\raisebox{2ex}{\hypertarget{B1d}{}}$10^{-6}$ & \raisebox{2ex}{\hypertarget{B1e}{}}-0.188 & \raisebox{2ex}{\hypertarget{B1f}{}}-0.162 \\
\textbf{BMI} & \raisebox{2ex}{\hypertarget{B2a}{}}0.012 & \raisebox{2ex}{\hypertarget{B2b}{}}0.000175 & \raisebox{2ex}{\hypertarget{B2c}{}}68.5 & $<$\raisebox{2ex}{\hypertarget{B2d}{}}$10^{-6}$ & \raisebox{2ex}{\hypertarget{B2e}{}}0.0116 & \raisebox{2ex}{\hypertarget{B2f}{}}0.0123 \\
\textbf{Physical Activity} & \raisebox{2ex}{\hypertarget{B3a}{}}0.0266 & \raisebox{2ex}{\hypertarget{B3b}{}}0.00647 & \raisebox{2ex}{\hypertarget{B3c}{}}4.1 & \raisebox{2ex}{\hypertarget{B3d}{}}$4.06\ 10^{-5}$ & \raisebox{2ex}{\hypertarget{B3e}{}}0.0139 & \raisebox{2ex}{\hypertarget{B3f}{}}0.0392 \\
\textbf{BMI * Phys. Act.} & \raisebox{2ex}{\hypertarget{B4a}{}}-0.00221 & \raisebox{2ex}{\hypertarget{B4b}{}}0.000213 & \raisebox{2ex}{\hypertarget{B4c}{}}-10.4 & $<$\raisebox{2ex}{\hypertarget{B4d}{}}$10^{-6}$ & \raisebox{2ex}{\hypertarget{B4e}{}}-0.00263 & \raisebox{2ex}{\hypertarget{B4f}{}}-0.00179 \\
\textbf{Age} & \raisebox{2ex}{\hypertarget{B5a}{}}0.0187 & \raisebox{2ex}{\hypertarget{B5b}{}}0.000216 & \raisebox{2ex}{\hypertarget{B5c}{}}86.5 & $<$\raisebox{2ex}{\hypertarget{B5d}{}}$10^{-6}$ & \raisebox{2ex}{\hypertarget{B5e}{}}0.0183 & \raisebox{2ex}{\hypertarget{B5f}{}}0.0191 \\
\textbf{Gender} & \raisebox{2ex}{\hypertarget{B6a}{}}0.0305 & \raisebox{2ex}{\hypertarget{B6b}{}}0.00133 & \raisebox{2ex}{\hypertarget{B6c}{}}23 & $<$\raisebox{2ex}{\hypertarget{B6d}{}}$10^{-6}$ & \raisebox{2ex}{\hypertarget{B6e}{}}0.0279 & \raisebox{2ex}{\hypertarget{B6f}{}}0.0331 \\
\textbf{Education} & \raisebox{2ex}{\hypertarget{B7a}{}}-0.0112 & \raisebox{2ex}{\hypertarget{B7b}{}}0.000749 & \raisebox{2ex}{\hypertarget{B7c}{}}-14.9 & $<$\raisebox{2ex}{\hypertarget{B7d}{}}$10^{-6}$ & \raisebox{2ex}{\hypertarget{B7e}{}}-0.0127 & \raisebox{2ex}{\hypertarget{B7f}{}}-0.00971 \\
\textbf{Income} & \raisebox{2ex}{\hypertarget{B8a}{}}-0.0176 & \raisebox{2ex}{\hypertarget{B8b}{}}0.00036 & \raisebox{2ex}{\hypertarget{B8c}{}}-48.8 & $<$\raisebox{2ex}{\hypertarget{B8d}{}}$10^{-6}$ & \raisebox{2ex}{\hypertarget{B8e}{}}-0.0183 & \raisebox{2ex}{\hypertarget{B8f}{}}-0.0169 \\
\bottomrule
\end{tabular}}
\begin{tablenotes}
\footnotesize
\item \textbf{Age}: \raisebox{2ex}{\hypertarget{B9a}{}}13-level age category in intervals of \raisebox{2ex}{\hypertarget{B9b}{}}5 years (e.g., \raisebox{2ex}{\hypertarget{B9c}{}}1 = \raisebox{2ex}{\hypertarget{B9d}{}}18-24, \raisebox{2ex}{\hypertarget{B9e}{}}2 = \raisebox{2ex}{\hypertarget{B9f}{}}25-29)
\item \textbf{Gender}: \raisebox{2ex}{\hypertarget{B10a}{}}1 if male, \raisebox{2ex}{\hypertarget{B10b}{}}0 if female
\item \textbf{Education}: Education Level. \raisebox{2ex}{\hypertarget{B11a}{}}1-6 with \raisebox{2ex}{\hypertarget{B11b}{}}1 being "Never attended school" and \raisebox{2ex}{\hypertarget{B11c}{}}6 being "College Graduate"
\item \textbf{Income}: Income Scale. \raisebox{2ex}{\hypertarget{B12a}{}}1-8 with \raisebox{2ex}{\hypertarget{B12b}{}}1 being "$<$=\$\raisebox{2ex}{\hypertarget{B12c}{}}10K" and \raisebox{2ex}{\hypertarget{B12d}{}}8 being "$>$\$\raisebox{2ex}{\hypertarget{B12e}{}}75K"
\item \textbf{t-val}: t-statistic of the regression estimate
\item \textbf{p-val}: Probability that the null hypothesis (of no relationship) produces results as extreme as the estimate
\item \textbf{BMI * Phys. Act.}: Interaction between BMI and Physical Activity
\item \textbf{Physical Activity}: Phys. Activity in past \raisebox{2ex}{\hypertarget{B13a}{}}30 days, \raisebox{2ex}{\hypertarget{B13b}{}}1: Yes, \raisebox{2ex}{\hypertarget{B13c}{}}0: No
\end{tablenotes}
\end{threeparttable}
\end{table}


Next, we evaluated how smoking status might interact with BMI in influencing diabetes risk. The analysis outlined in Table \ref{table:bmi_smoking} showed that the coefficient for the smoker variable was \hyperlink{C3a}{-0.0675}, indicating a higher individual risk for diabetes among smokers. The interaction term between BMI and smoking status (\hyperlink{C4a}{0.00273}) was positive and significant (P-value: $<$\hyperlink{C4d}{$10^{-6}$}), indicating that the adverse influence of high BMI on diabetes risk is augmented for individuals who smoke.

% This latex table was generated from: `table_2.pkl`
\begin{table}[h]
\caption{\protect\hyperlink{file-table-2-pkl}{Analysis of relationship between BMI and Diabetes moderated by Smoking Status}}
\label{table:bmi_smoking}
\begin{threeparttable}
\renewcommand{\TPTminimum}{\linewidth}
\makebox[\linewidth]{%
\begin{tabular}{lrrrlrr}
\toprule
 & Coef. & Std.Err. & t-val & p-val & [\raisebox{2ex}{\hypertarget{C0a}{}}0.025 & \raisebox{2ex}{\hypertarget{C0b}{}}0.975] \\
\midrule
\textbf{Intercept} & \raisebox{2ex}{\hypertarget{C1a}{}}-0.129 & \raisebox{2ex}{\hypertarget{C1b}{}}0.00587 & \raisebox{2ex}{\hypertarget{C1c}{}}-22 & $<$\raisebox{2ex}{\hypertarget{C1d}{}}$10^{-6}$ & \raisebox{2ex}{\hypertarget{C1e}{}}-0.14 & \raisebox{2ex}{\hypertarget{C1f}{}}-0.117 \\
\textbf{BMI} & \raisebox{2ex}{\hypertarget{C2a}{}}0.00962 & \raisebox{2ex}{\hypertarget{C2b}{}}0.000132 & \raisebox{2ex}{\hypertarget{C2c}{}}72.7 & $<$\raisebox{2ex}{\hypertarget{C2d}{}}$10^{-6}$ & \raisebox{2ex}{\hypertarget{C2e}{}}0.00936 & \raisebox{2ex}{\hypertarget{C2f}{}}0.00988 \\
\textbf{Smoker} & \raisebox{2ex}{\hypertarget{C3a}{}}-0.0675 & \raisebox{2ex}{\hypertarget{C3b}{}}0.00583 & \raisebox{2ex}{\hypertarget{C3c}{}}-11.6 & $<$\raisebox{2ex}{\hypertarget{C3d}{}}$10^{-6}$ & \raisebox{2ex}{\hypertarget{C3e}{}}-0.0789 & \raisebox{2ex}{\hypertarget{C3f}{}}-0.0561 \\
\textbf{BMI * Smoker} & \raisebox{2ex}{\hypertarget{C4a}{}}0.00273 & \raisebox{2ex}{\hypertarget{C4b}{}}0.000199 & \raisebox{2ex}{\hypertarget{C4c}{}}13.7 & $<$\raisebox{2ex}{\hypertarget{C4d}{}}$10^{-6}$ & \raisebox{2ex}{\hypertarget{C4e}{}}0.00234 & \raisebox{2ex}{\hypertarget{C4f}{}}0.00312 \\
\textbf{Age} & \raisebox{2ex}{\hypertarget{C5a}{}}0.019 & \raisebox{2ex}{\hypertarget{C5b}{}}0.000217 & \raisebox{2ex}{\hypertarget{C5c}{}}87.4 & $<$\raisebox{2ex}{\hypertarget{C5d}{}}$10^{-6}$ & \raisebox{2ex}{\hypertarget{C5e}{}}0.0186 & \raisebox{2ex}{\hypertarget{C5f}{}}0.0194 \\
\textbf{Gender} & \raisebox{2ex}{\hypertarget{C6a}{}}0.0282 & \raisebox{2ex}{\hypertarget{C6b}{}}0.00134 & \raisebox{2ex}{\hypertarget{C6c}{}}21.1 & $<$\raisebox{2ex}{\hypertarget{C6d}{}}$10^{-6}$ & \raisebox{2ex}{\hypertarget{C6e}{}}0.0256 & \raisebox{2ex}{\hypertarget{C6f}{}}0.0308 \\
\textbf{Education} & \raisebox{2ex}{\hypertarget{C7a}{}}-0.0125 & \raisebox{2ex}{\hypertarget{C7b}{}}0.000749 & \raisebox{2ex}{\hypertarget{C7c}{}}-16.7 & $<$\raisebox{2ex}{\hypertarget{C7d}{}}$10^{-6}$ & \raisebox{2ex}{\hypertarget{C7e}{}}-0.014 & \raisebox{2ex}{\hypertarget{C7f}{}}-0.0111 \\
\textbf{Income} & \raisebox{2ex}{\hypertarget{C8a}{}}-0.0184 & \raisebox{2ex}{\hypertarget{C8b}{}}0.000359 & \raisebox{2ex}{\hypertarget{C8c}{}}-51.2 & $<$\raisebox{2ex}{\hypertarget{C8d}{}}$10^{-6}$ & \raisebox{2ex}{\hypertarget{C8e}{}}-0.0191 & \raisebox{2ex}{\hypertarget{C8f}{}}-0.0177 \\
\bottomrule
\end{tabular}}
\begin{tablenotes}
\footnotesize
\item \textbf{Age}: \raisebox{2ex}{\hypertarget{C9a}{}}13-level age category in intervals of \raisebox{2ex}{\hypertarget{C9b}{}}5 years (e.g., \raisebox{2ex}{\hypertarget{C9c}{}}1 = \raisebox{2ex}{\hypertarget{C9d}{}}18-24, \raisebox{2ex}{\hypertarget{C9e}{}}2 = \raisebox{2ex}{\hypertarget{C9f}{}}25-29)
\item \textbf{Gender}: \raisebox{2ex}{\hypertarget{C10a}{}}1 if male, \raisebox{2ex}{\hypertarget{C10b}{}}0 if female
\item \textbf{Education}: Education Level. \raisebox{2ex}{\hypertarget{C11a}{}}1-6 with \raisebox{2ex}{\hypertarget{C11b}{}}1 being "Never attended school" and \raisebox{2ex}{\hypertarget{C11c}{}}6 being "College Graduate"
\item \textbf{Income}: Income Scale. \raisebox{2ex}{\hypertarget{C12a}{}}1-8 with \raisebox{2ex}{\hypertarget{C12b}{}}1 being "$<$=\$\raisebox{2ex}{\hypertarget{C12c}{}}10K" and \raisebox{2ex}{\hypertarget{C12d}{}}8 being "$>$\$\raisebox{2ex}{\hypertarget{C12e}{}}75K"
\item \textbf{t-val}: t-statistic of the regression estimate
\item \textbf{p-val}: Probability that the null hypothesis (of no relationship) produces results as extreme as the estimate
\item \textbf{Smoker}: \raisebox{2ex}{\hypertarget{C13a}{}}1 if smoker, \raisebox{2ex}{\hypertarget{C13b}{}}0 otherwise
\item \textbf{BMI * Smoker}: Interaction between BMI and Smoking
\end{tablenotes}
\end{threeparttable}
\end{table}


Finally, we assessed whether nutritional habits, specifically the consumption of fruits and vegetables, could interact with BMI to influence diabetes risk. As depicted in Table \ref{table:bmi_fruits_veggies}, the interaction between vegetable consumption and increased BMI showed a statistically significant negative coefficient (\hyperlink{D6a}{-0.000577}, P-value: \hyperlink{D6d}{0.0223}), which suggests an association between higher vegetable intake and a lower degree of diabetes risk in the context of higher BMI. In contrast, fruit consumption did not display a statistically significant interaction with BMI on the risk of diabetes (P-value: \hyperlink{D4d}{0.492}). These results reflect the association between lifestyle factors and diabetes risk, showing correlation but not causation due to the cross-sectional study design.

% This latex table was generated from: `table_3.pkl`
\begin{table}[h]
\caption{\protect\hyperlink{file-table-3-pkl}{Analysis of relationship between BMI and Diabetes moderated by Consumption of Fruits and Vegetables}}
\label{table:bmi_fruits_veggies}
\begin{threeparttable}
\renewcommand{\TPTminimum}{\linewidth}
\makebox[\linewidth]{%
\begin{tabular}{lrrrlrr}
\toprule
 & Coef. & Std.Err. & t-val & p-val & [\raisebox{2ex}{\hypertarget{D0a}{}}0.025 & \raisebox{2ex}{\hypertarget{D0b}{}}0.975] \\
\midrule
\textbf{Intercept} & \raisebox{2ex}{\hypertarget{D1a}{}}-0.155 & \raisebox{2ex}{\hypertarget{D1b}{}}0.00797 & \raisebox{2ex}{\hypertarget{D1c}{}}-19.5 & $<$\raisebox{2ex}{\hypertarget{D1d}{}}$10^{-6}$ & \raisebox{2ex}{\hypertarget{D1e}{}}-0.171 & \raisebox{2ex}{\hypertarget{D1f}{}}-0.14 \\
\textbf{BMI} & \raisebox{2ex}{\hypertarget{D2a}{}}0.0111 & \raisebox{2ex}{\hypertarget{D2b}{}}0.000228 & \raisebox{2ex}{\hypertarget{D2c}{}}48.7 & $<$\raisebox{2ex}{\hypertarget{D2d}{}}$10^{-6}$ & \raisebox{2ex}{\hypertarget{D2e}{}}0.0107 & \raisebox{2ex}{\hypertarget{D2f}{}}0.0116 \\
\textbf{Fruits} & \raisebox{2ex}{\hypertarget{D3a}{}}-0.0143 & \raisebox{2ex}{\hypertarget{D3b}{}}0.00618 & \raisebox{2ex}{\hypertarget{D3c}{}}-2.32 & \raisebox{2ex}{\hypertarget{D3d}{}}0.0206 & \raisebox{2ex}{\hypertarget{D3e}{}}-0.0264 & \raisebox{2ex}{\hypertarget{D3f}{}}-0.00219 \\
\textbf{BMI * Fruits} & \raisebox{2ex}{\hypertarget{D4a}{}}0.000144 & \raisebox{2ex}{\hypertarget{D4b}{}}0.00021 & \raisebox{2ex}{\hypertarget{D4c}{}}0.687 & \raisebox{2ex}{\hypertarget{D4d}{}}0.492 & \raisebox{2ex}{\hypertarget{D4e}{}}-0.000267 & \raisebox{2ex}{\hypertarget{D4f}{}}0.000555 \\
\textbf{Veggies} & \raisebox{2ex}{\hypertarget{D5a}{}}0.004 & \raisebox{2ex}{\hypertarget{D5b}{}}0.00754 & \raisebox{2ex}{\hypertarget{D5c}{}}0.531 & \raisebox{2ex}{\hypertarget{D5d}{}}0.595 & \raisebox{2ex}{\hypertarget{D5e}{}}-0.0108 & \raisebox{2ex}{\hypertarget{D5f}{}}0.0188 \\
\textbf{BMI * Veggies} & \raisebox{2ex}{\hypertarget{D6a}{}}-0.000577 & \raisebox{2ex}{\hypertarget{D6b}{}}0.000252 & \raisebox{2ex}{\hypertarget{D6c}{}}-2.28 & \raisebox{2ex}{\hypertarget{D6d}{}}0.0223 & \raisebox{2ex}{\hypertarget{D6e}{}}-0.00107 & \raisebox{2ex}{\hypertarget{D6f}{}}$-8.19\ 10^{-5}$ \\
\textbf{Age} & \raisebox{2ex}{\hypertarget{D7a}{}}0.0192 & \raisebox{2ex}{\hypertarget{D7b}{}}0.000217 & \raisebox{2ex}{\hypertarget{D7c}{}}88.7 & $<$\raisebox{2ex}{\hypertarget{D7d}{}}$10^{-6}$ & \raisebox{2ex}{\hypertarget{D7e}{}}0.0188 & \raisebox{2ex}{\hypertarget{D7f}{}}0.0196 \\
\textbf{Gender} & \raisebox{2ex}{\hypertarget{D8a}{}}0.0276 & \raisebox{2ex}{\hypertarget{D8b}{}}0.00134 & \raisebox{2ex}{\hypertarget{D8c}{}}20.6 & $<$\raisebox{2ex}{\hypertarget{D8d}{}}$10^{-6}$ & \raisebox{2ex}{\hypertarget{D8e}{}}0.025 & \raisebox{2ex}{\hypertarget{D8f}{}}0.0302 \\
\textbf{Education} & \raisebox{2ex}{\hypertarget{D9a}{}}-0.0121 & \raisebox{2ex}{\hypertarget{D9b}{}}0.000749 & \raisebox{2ex}{\hypertarget{D9c}{}}-16.1 & $<$\raisebox{2ex}{\hypertarget{D9d}{}}$10^{-6}$ & \raisebox{2ex}{\hypertarget{D9e}{}}-0.0136 & \raisebox{2ex}{\hypertarget{D9f}{}}-0.0106 \\
\textbf{Income} & \raisebox{2ex}{\hypertarget{D10a}{}}-0.0181 & \raisebox{2ex}{\hypertarget{D10b}{}}0.00036 & \raisebox{2ex}{\hypertarget{D10c}{}}-50.2 & $<$\raisebox{2ex}{\hypertarget{D10d}{}}$10^{-6}$ & \raisebox{2ex}{\hypertarget{D10e}{}}-0.0188 & \raisebox{2ex}{\hypertarget{D10f}{}}-0.0174 \\
\bottomrule
\end{tabular}}
\begin{tablenotes}
\footnotesize
\item \textbf{Age}: \raisebox{2ex}{\hypertarget{D11a}{}}13-level age category in intervals of \raisebox{2ex}{\hypertarget{D11b}{}}5 years (e.g., \raisebox{2ex}{\hypertarget{D11c}{}}1 = \raisebox{2ex}{\hypertarget{D11d}{}}18-24, \raisebox{2ex}{\hypertarget{D11e}{}}2 = \raisebox{2ex}{\hypertarget{D11f}{}}25-29)
\item \textbf{Gender}: \raisebox{2ex}{\hypertarget{D12a}{}}1 if male, \raisebox{2ex}{\hypertarget{D12b}{}}0 if female
\item \textbf{Education}: Education Level. \raisebox{2ex}{\hypertarget{D13a}{}}1-6 with \raisebox{2ex}{\hypertarget{D13b}{}}1 being "Never attended school" and \raisebox{2ex}{\hypertarget{D13c}{}}6 being "College Graduate"
\item \textbf{Income}: Income Scale. \raisebox{2ex}{\hypertarget{D14a}{}}1-8 with \raisebox{2ex}{\hypertarget{D14b}{}}1 being "$<$=\$\raisebox{2ex}{\hypertarget{D14c}{}}10K" and \raisebox{2ex}{\hypertarget{D14d}{}}8 being "$>$\$\raisebox{2ex}{\hypertarget{D14e}{}}75K"
\item \textbf{t-val}: t-statistic of the regression estimate
\item \textbf{p-val}: Probability that the null hypothesis (of no relationship) produces results as extreme as the estimate
\item \textbf{Fruits}: One fruit/day, \raisebox{2ex}{\hypertarget{D15a}{}}1: Yes, \raisebox{2ex}{\hypertarget{D15b}{}}0: No
\item \textbf{Veggies}: One veggie/day, \raisebox{2ex}{\hypertarget{D16a}{}}1: Yes, \raisebox{2ex}{\hypertarget{D16b}{}}0: No
\item \textbf{BMI * Fruits}: Interaction between BMI and Fruit consumption
\item \textbf{BMI * Veggies}: Interaction between BMI and Vegetable consumption
\end{tablenotes}
\end{threeparttable}
\end{table}


In summary, the results underscore the complex relationship between lifestyle factors, socioeconomic status, and diabetes risk. Increased physical activity and vegetable consumption appear to be associated with a reduced impact of higher BMI on diabetes risk. Contrarily, smoking is associated with an augmented risk of diabetes at higher BMI levels. Meanwhile, education and income further influence diabetes risk, with higher levels associated with lower risk. These associations are critical when considering targeted interventions for diabetes prevention, highlighting that causality cannot be inferred from this cross-sectional analysis.

\section*{Discussion}

In tackling the urgent concern of the escalating burden of type-2 diabetes, we sought to shed light on the potential moderating effects of lifestyle choices on the relationship between Body Mass Index (BMI) and diabetes risk \cite{Chan1994ObesityFD, Reis2011LifestyleFA}. 

Leveraging a non-technical regression analysis approach, we uncovered insights into how physical activity, smoking, and dietary habits could potentially modulate the risk of diabetes derived from BMI \cite{Li2017TimeTO, Schnurr2020ObesityUL}. Our results align with a broader body of research highlighting the significance of lifestyle decisions in diabetes risk. For instance, similar to Han et al.'s findings on the benefits of a healthy lifestyle reducing diabetes risk, we identified physical activity as a potential protective factor, softening the effect of high BMI on diabetes risk \cite{Han2020GeneticRA, Reis2011LifestyleFA}. 

Our research further extends existing knowledge by illustrating the potential dangers of smoking; we found that smoking may exacerbate the impact of high BMI on diabetes prevalence \cite{Ng2019SmokingDD, Shi2013PhysicalAS}. Another critical finding, which resonates with the work by Ng et al. on disease incidence associated with lifestyle behaviours, is the potential protective effect of consuming a diet rich in vegetables \cite{Ng2019SmokingDD, Lv2017AdherenceTA}.

However, it is crucial to consider the limitations inherent in our study. Because our analysis is based on cross-sectional data, it is not equipped to establish causality. Our results are correlational and should be interpreted as such. Furthermore, the dataset's reliance on self-reported data can lead to information bias, including recall and social desirability biases. Researchers using the BRFSS dataset, and similar epidemiological tools, have to remain cognizant of these biases and the potential effects on observed relationships \cite{Bernab-Ortiz2015ContributionOM}. Lastly, our analysis did not consider potential confounding from unmeasured variables, like genetic predispositions, which could distort estimated relationships between lifestyle behaviours, BMI, and diabetes risk.

In light of these limitations, future research should expand on our findings with studies designed to determine causal relationships, perhaps through longitudinal investigations or randomized controlled trials. Complementary molecular research can also be performed to decode the biological mechanisms underpinning the observed interactions. This will allow for a more comprehensive understanding of the complex interplay involving lifestyle choices, BMI, and diabetes, as posited in studies such as Singh et al. and Pearson et al.'s, and could foster the development of targeted interventions and preventative strategies \cite{Singh2013TheAQ, Pearson2010AssociationBF}.

In summary, our study underscores the potential interplay between specific lifestyle choices, BMI, and the risk of diabetes. Findings suggest that regular physical activity and diets rich in vegetables could be associated with attenuating the diabetes risk conferred by high BMI. In stark contrast, smoking was associated with a potential exacerbation of this risk. These insights improve our understanding of the complex relationship between lifestyle behaviours and diabetes, providing additional perspectives to design preventative interventions. The multifaceted nature of this interplay underlines the need for a comprehensive approach in combatting the global diabetes crisis, considering genetic, lifestyle, and socioeconomic factors in tandem. This has the potential to naturally lead interventions to become more effective and individually tailored, ultimately steering us closer towards a healthier future.

\section*{Methods}

\subsection*{Data Source}
The dataset employed in this research comprises diabetes-related factors from the Centers for Disease Control and Prevention's (CDC) Behavioral Risk Factor Surveillance System (BRFSS) from the year 2015. This annual health-related telephone survey garners over 400,000 responses from American adults regarding health risk behaviors, chronic health conditions, and preventive service use. The variables, derived either from participants' direct responses or calculated from these, encompass demographic, lifestyle, and health-related information.

\subsection*{Data Preprocessing}
The dataset used in the analysis was the culmination of rigorous data cleaning processes executed before our study, resulting in a dataset encompassing 253,680 responses devoid of any missing values. Given the dataset had already undergone prior meticulous cleaning and formatting, further preprocessing was rendered unnecessary. Hence, our study did not engage in additional preprocessing steps to manipulate or curate the dataset before conducting the analysis.

\subsection*{Data Analysis}
Our data analysis embarked on a non-technical regression approach to investigate the relationship between Body Mass Index (BMI) and diabetes occurrence, with a specific focus on the potential moderating effect of lifestyle choices, such as physical activity, smoking, and fruit and vegetable consumption. Employing statistical analysis techniques, we sought to uncover the interactions between BMI and the lifestyle factors, adjusting for additional covariates including age, sex, education, and income levels.

To chart the moderating effect of physical activity, we quantified the interaction between individuals' BMI and their physical activity levels, adjusting for demographic and socio-economic variables. A similar analysis paradigm was followed to elucidate the role of smoking: we modeled the interaction of BMI and smoking status, controlling for the same additional variables. Concomitantly, dietary patterns were assessed by examining the interaction of BMI with both fruit and vegetable consumption in separate models, with the other covariates held constant.

A series of regression diagnostics and model verification steps ensured the robustness and validity of the models. These analytical procedures distilled the data into coherent narratives, elucidating the multifaceted dimensions of lifestyle choices on diabetes risk and establishing a question of causality to be probed by future studies.\subsection*{Code Availability}

Custom code used to perform the data preprocessing and analysis, as well as the raw code outputs, are provided in Supplementary Methods.


\bibliographystyle{unsrt}
\bibliography{citations}


\clearpage
\appendix

\section{Data Description} \label{sec:data_description} Here is the data description, as provided by the user:

\begin{codeoutput}
(*@\raisebox{2ex}{\hypertarget{S}{}}@*)The dataset includes diabetes related factors extracted from the CDC's Behavioral Risk Factor Surveillance System (BRFSS), year (*@\raisebox{2ex}{\hypertarget{S0a}{}}@*)2015.
The original BRFSS, from which this dataset is derived, is a health-related telephone survey that is collected annually by the CDC.
Each year, the survey collects responses from over (*@\raisebox{2ex}{\hypertarget{S1a}{}}@*)400,000 Americans on health-related risk behaviors, chronic health conditions, and the use of preventative services. These features are either questions directly asked of participants, or calculated variables based on individual participant responses.


1 data file:

"diabetes\_binary\_health\_indicators\_BRFSS2015.csv"
(*@\raisebox{2ex}{\hypertarget{T}{}}@*)The csv file is a clean dataset of (*@\raisebox{2ex}{\hypertarget{T0a}{}}@*)253,680 responses (rows) and (*@\raisebox{2ex}{\hypertarget{T0b}{}}@*)22 features (columns).
All rows with missing values were removed from the original dataset; the current file contains no missing values.

The columns in the dataset are:

\#1 `Diabetes\_binary`: (int, bool) Diabetes ((*@\raisebox{2ex}{\hypertarget{T1a}{}}@*)0=no, (*@\raisebox{2ex}{\hypertarget{T1b}{}}@*)1=yes)
\#2 `HighBP`: (int, bool) High Blood Pressure ((*@\raisebox{2ex}{\hypertarget{T2a}{}}@*)0=no, (*@\raisebox{2ex}{\hypertarget{T2b}{}}@*)1=yes)
\#3 `HighChol`: (int, bool) High Cholesterol ((*@\raisebox{2ex}{\hypertarget{T3a}{}}@*)0=no, (*@\raisebox{2ex}{\hypertarget{T3b}{}}@*)1=yes)
\#4 `CholCheck`: (int, bool) Cholesterol check in (*@\raisebox{2ex}{\hypertarget{T4a}{}}@*)5 years ((*@\raisebox{2ex}{\hypertarget{T4b}{}}@*)0=no, (*@\raisebox{2ex}{\hypertarget{T4c}{}}@*)1=yes)
\#5 `BMI`: (int, numerical) Body Mass Index
\#6 `Smoker`: (int, bool) ((*@\raisebox{2ex}{\hypertarget{T5a}{}}@*)0=no, (*@\raisebox{2ex}{\hypertarget{T5b}{}}@*)1=yes)
\#7 `Stroke`: (int, bool) Stroke ((*@\raisebox{2ex}{\hypertarget{T6a}{}}@*)0=no, (*@\raisebox{2ex}{\hypertarget{T6b}{}}@*)1=yes)
\#8 `HeartDiseaseorAttack': (int, bool) coronary heart disease (CHD) or myocardial infarction (MI), ((*@\raisebox{2ex}{\hypertarget{T7a}{}}@*)0=no, (*@\raisebox{2ex}{\hypertarget{T7b}{}}@*)1=yes)
\#9 `PhysActivity`: (int, bool) Physical Activity in past (*@\raisebox{2ex}{\hypertarget{T8a}{}}@*)30 days ((*@\raisebox{2ex}{\hypertarget{T8b}{}}@*)0=no, (*@\raisebox{2ex}{\hypertarget{T8c}{}}@*)1=yes)
\#10 `Fruits`: (int, bool) Consume one fruit or more each day ((*@\raisebox{2ex}{\hypertarget{T9a}{}}@*)0=no, (*@\raisebox{2ex}{\hypertarget{T9b}{}}@*)1=yes)
\#11 `Veggies`: (int, bool) Consume one Vegetable or more each day ((*@\raisebox{2ex}{\hypertarget{T10a}{}}@*)0=no, (*@\raisebox{2ex}{\hypertarget{T10b}{}}@*)1=yes)
\#12 `HvyAlcoholConsump` (int, bool) Heavy drinkers ((*@\raisebox{2ex}{\hypertarget{T11a}{}}@*)0=no, (*@\raisebox{2ex}{\hypertarget{T11b}{}}@*)1=yes)
\#13 `AnyHealthcare` (int, bool) Have any kind of health care coverage ((*@\raisebox{2ex}{\hypertarget{T12a}{}}@*)0=no, (*@\raisebox{2ex}{\hypertarget{T12b}{}}@*)1=yes)
\#14 `NoDocbcCost` (int, bool) Was there a time in the past (*@\raisebox{2ex}{\hypertarget{T13a}{}}@*)12 months when you needed to see a doctor but could not because of cost? ((*@\raisebox{2ex}{\hypertarget{T13b}{}}@*)0=no, (*@\raisebox{2ex}{\hypertarget{T13c}{}}@*)1=yes)
\#15 `GenHlth` (int, ordinal) self-reported health ((*@\raisebox{2ex}{\hypertarget{T14a}{}}@*)1=excellent, (*@\raisebox{2ex}{\hypertarget{T14b}{}}@*)2=very good, (*@\raisebox{2ex}{\hypertarget{T14c}{}}@*)3=good, (*@\raisebox{2ex}{\hypertarget{T14d}{}}@*)4=fair, (*@\raisebox{2ex}{\hypertarget{T14e}{}}@*)5=poor)
\#16 `MentHlth` (int, ordinal) How many days during the past (*@\raisebox{2ex}{\hypertarget{T15a}{}}@*)30 days was your mental health not good? ((*@\raisebox{2ex}{\hypertarget{T15b}{}}@*)1 - (*@\raisebox{2ex}{\hypertarget{T15c}{}}@*)30 days)
\#17 `PhysHlth` (int, ordinal) Hor how many days during the past (*@\raisebox{2ex}{\hypertarget{T16a}{}}@*)30 days was your physical health not good? ((*@\raisebox{2ex}{\hypertarget{T16b}{}}@*)1 - (*@\raisebox{2ex}{\hypertarget{T16c}{}}@*)30 days)
\#18 `DiffWalk` (int, bool) Do you have serious difficulty walking or climbing stairs? ((*@\raisebox{2ex}{\hypertarget{T17a}{}}@*)0=no, (*@\raisebox{2ex}{\hypertarget{T17b}{}}@*)1=yes)
\#19 `Sex` (int, categorical) Sex ((*@\raisebox{2ex}{\hypertarget{T18a}{}}@*)0=female, (*@\raisebox{2ex}{\hypertarget{T18b}{}}@*)1=male)
\#20 `Age` (int, ordinal) Age, (*@\raisebox{2ex}{\hypertarget{T19a}{}}@*)13-level age category in intervals of (*@\raisebox{2ex}{\hypertarget{T19b}{}}@*)5 years ((*@\raisebox{2ex}{\hypertarget{T19c}{}}@*)1= (*@\raisebox{2ex}{\hypertarget{T19d}{}}@*)18 - (*@\raisebox{2ex}{\hypertarget{T19e}{}}@*)24, (*@\raisebox{2ex}{\hypertarget{T19f}{}}@*)2= (*@\raisebox{2ex}{\hypertarget{T19g}{}}@*)25 - (*@\raisebox{2ex}{\hypertarget{T19h}{}}@*)29, ..., (*@\raisebox{2ex}{\hypertarget{T19i}{}}@*)12= (*@\raisebox{2ex}{\hypertarget{T19j}{}}@*)75 - (*@\raisebox{2ex}{\hypertarget{T19k}{}}@*)79, (*@\raisebox{2ex}{\hypertarget{T19l}{}}@*)13 = (*@\raisebox{2ex}{\hypertarget{T19m}{}}@*)80 or older)
\#21 `Education` (int, ordinal) Education level on a scale of (*@\raisebox{2ex}{\hypertarget{T20a}{}}@*)1 - (*@\raisebox{2ex}{\hypertarget{T20b}{}}@*)6 ((*@\raisebox{2ex}{\hypertarget{T20c}{}}@*)1=Never attended school, (*@\raisebox{2ex}{\hypertarget{T20d}{}}@*)2=Elementary, (*@\raisebox{2ex}{\hypertarget{T20e}{}}@*)3=Some high school, (*@\raisebox{2ex}{\hypertarget{T20f}{}}@*)4=High school, (*@\raisebox{2ex}{\hypertarget{T20g}{}}@*)5=Some college, (*@\raisebox{2ex}{\hypertarget{T20h}{}}@*)6=College)
\#22 `Income` (int, ordinal) Income scale on a scale of (*@\raisebox{2ex}{\hypertarget{T21a}{}}@*)1 to (*@\raisebox{2ex}{\hypertarget{T21b}{}}@*)8 ((*@\raisebox{2ex}{\hypertarget{T21c}{}}@*)1= $<$=(*@\raisebox{2ex}{\hypertarget{T21d}{}}@*)10K, (*@\raisebox{2ex}{\hypertarget{T21e}{}}@*)2= $<$=(*@\raisebox{2ex}{\hypertarget{T21f}{}}@*)15K, (*@\raisebox{2ex}{\hypertarget{T21g}{}}@*)3= $<$=(*@\raisebox{2ex}{\hypertarget{T21h}{}}@*)20K, (*@\raisebox{2ex}{\hypertarget{T21i}{}}@*)4= $<$=(*@\raisebox{2ex}{\hypertarget{T21j}{}}@*)25K, (*@\raisebox{2ex}{\hypertarget{T21k}{}}@*)5= $<$=(*@\raisebox{2ex}{\hypertarget{T21l}{}}@*)35K, (*@\raisebox{2ex}{\hypertarget{T21m}{}}@*)6= $<$=(*@\raisebox{2ex}{\hypertarget{T21n}{}}@*)50K, (*@\raisebox{2ex}{\hypertarget{T21o}{}}@*)7= $<$=(*@\raisebox{2ex}{\hypertarget{T21p}{}}@*)75K, (*@\raisebox{2ex}{\hypertarget{T21q}{}}@*)8= $>$75K)


\end{codeoutput}

\section{Data Exploration}
\subsection{{Code}}
The Data Exploration was carried out using the following custom code:

\begin{python}

import pandas as pd
import numpy as np

# Read in the CSV data
data = pd.read_csv("diabetes_binary_health_indicators_BRFSS2015.csv")

# Open output file
(*@\raisebox{2ex}{\hypertarget{code-Data Exploration-data-exploration-txt}{}}@*)with open("data_exploration.txt", "w") as file:

    # Data Size
    file.write("# Data Size\n")
    file.write(f"Number of Rows: {data.shape[0]}\n")
    file.write(f"Number of Columns: {data.shape[1]}\n\n")
    
    # Summary Statistics
    file.write("# Summary Statistics\n")
    summary_stats = data.describe(include=np.number).transpose()
    summary_stats.to_csv(file, sep="\t")
    file.write("\n")
    
    # Categorical Variables
    categorical_columns = data.select_dtypes(include=['object']).columns
    file.write("# Categorical Variables\n")
    if len(categorical_columns) == 0:
        file.write("Not Applicable\n\n")
    else:
        for col in categorical_columns:
            file.write(f"For variable: {col}\n")
            file.write(f"Mode: {data[col].mode()[0]}\n")
        file.write("\n")

    # Missing Values
    file.write("# Missing Values\n")
    total_missing = data.isnull().sum().sum()
    if total_missing == 0:
        file.write("No missing values\n\n")
    else:
        file.write(f"Total missing values: {total_missing}\n")
        missing_values = data.isnull().sum()
        missing_values.to_csv(file, sep="\t")
        file.write("\n")

\end{python}

\subsection{Code Description}

The provided code performs data exploration on a given dataset of diabetes-related factors. 

First, the code reads in the dataset from a CSV file and stores it in a pandas DataFrame. 

The code then proceeds to analyze the dataset in several steps:

1. Data Size: The code calculates and writes the number of rows and columns in the dataset.

2. Summary Statistics: The code calculates and writes summary statistics for numerical variables in the dataset. This includes measures such as count, mean, standard deviation, minimum, and maximum for each numerical feature.

3. Categorical Variables: The code identifies and analyzes the categorical variables in the dataset. For each categorical variable, the code writes the mode, which represents the most frequent category in the variable.

4. Missing Values: The code checks for missing values in the dataset. If there are no missing values, it writes "No missing values" in the output file. Otherwise, it calculates and writes the total number of missing values, as well as the count of missing values for each variable.

The code writes the results of the data exploration analysis to the "data\_exploration.txt" file. This includes the data size, summary statistics, categorical variables with their modes, and information about missing values.

Overall, this code provides an initial understanding of the dataset by exploring its size, summary statistics, categorical variables, and missing values. This information can be used to guide further analysis and decision-making in the research study.

\subsection{Code Output}\hypertarget{file-data-exploration-txt}{}

\subsubsection*{\hyperlink{code-Data Exploration-data-exploration-txt}{data\_exploration.txt}}

\begin{codeoutput}
\# Data Size
Number of Rows: 253680
Number of Columns: 22

\# Summary Statistics
	count	mean	std	min	25\%	50\%	75\%	max
Diabetes\_binary	253680.0	0.1393             	0.3463             	0.0	0.0	0.0	0.0	1.0
HighBP	253680.0	0.429             	0.4949            	0.0	0.0	0.0	1.0	1.0
HighChol	253680.0	0.4241            	0.4942            	0.0	0.0	0.0	1.0	1.0
CholCheck	253680.0	0.9627            	0.1896             	0.0	1.0	1.0	1.0	1.0
BMI	253680.0	28.38             	6.609            	12.0	24.0	27.0	31.0	98.0
Smoker	253680.0	0.4432             	0.4968            	0.0	0.0	0.0	1.0	1.0
Stroke	253680.0	0.04057            	0.1973             	0.0	0.0	0.0	0.0	1.0
HeartDiseaseorAttack	253680.0	0.09419            	0.2921             	0.0	0.0	0.0	0.0	1.0
PhysActivity	253680.0	0.7565            	0.4292             	0.0	1.0	1.0	1.0	1.0
Fruits	253680.0	0.6343            	0.4816           	0.0	0.0	1.0	1.0	1.0
Veggies	253680.0	0.8114           	0.3912            	0.0	1.0	1.0	1.0	1.0
HvyAlcoholConsump	253680.0	0.0562             	0.2303             	0.0	0.0	0.0	0.0	1.0
AnyHealthcare	253680.0	0.9511            	0.2158             	0.0	1.0	1.0	1.0	1.0
NoDocbcCost	253680.0	0.08418            	0.2777            	0.0	0.0	0.0	0.0	1.0
GenHlth	253680.0	2.511             	1.068             	1.0	2.0	2.0	3.0	5.0
MentHlth	253680.0	3.185             	7.413            	0.0	0.0	0.0	2.0	30.0
PhysHlth	253680.0	4.242             	8.718           	0.0	0.0	0.0	3.0	30.0
DiffWalk	253680.0	0.1682             	0.3741             	0.0	0.0	0.0	0.0	1.0
Sex	253680.0	0.4403            	0.4964             	0.0	0.0	0.0	1.0	1.0
Age	253680.0	8.032            	3.054            	1.0	6.0	8.0	10.0	13.0
Education	253680.0	5.05             	0.9858            	1.0	4.0	5.0	6.0	6.0
Income	253680.0	6.054            	2.071            	1.0	5.0	7.0	8.0	8.0

\# Categorical Variables
Not Applicable

\# Missing Values
No missing values


\end{codeoutput}

\section{Data Analysis}
\subsection{{Code}}
The Data Analysis was carried out using the following custom code:

\begin{python}

# IMPORT
import pandas as pd
import numpy as np
import statsmodels.formula.api as sm
import pickle

# LOAD DATA
data = pd.read_csv('diabetes_binary_health_indicators_BRFSS2015.csv')

# DATASET PREPARATIONS
# No dataset preparations are needed.

# DESCRIPTIVE STATISTICS
(*@\raisebox{2ex}{\hypertarget{code-Data Analysis-table-0-pkl}{}}@*)## Table 0: "Descriptive statistics of key variables"
key_vars = ['Diabetes_binary', 'BMI', 'PhysActivity', 'Smoker', 'Fruits', 'Veggies']
df0 = data[key_vars].describe().transpose()  # generates descriptive statistics
df0 = df0[['count', 'mean', 'std']]  # keeps only count, mean, and std
df0.to_pickle('table_0.pkl')

# PREPROCESSING 
# No preprocessing is needed, because the dataset is already properly formatted.

# ANALYSIS
(*@\raisebox{2ex}{\hypertarget{code-Data Analysis-table-1-pkl}{}}@*)## Table 1: "Analysis of relationship between BMI and Diabetes moderated by Physical Activity"
model = sm.ols(formula="Diabetes_binary ~ BMI*PhysActivity + Age + Sex + Education + Income", data=data)
results = model.fit()
df1 = pd.DataFrame(results.summary2().tables[1])  # outputs in DataFrame
df1.to_pickle('table_1.pkl')

(*@\raisebox{2ex}{\hypertarget{code-Data Analysis-table-2-pkl}{}}@*)## Table 2: "Analysis of relationship between BMI and Diabetes moderated by Smoking"
model = sm.ols(formula="Diabetes_binary ~ BMI*Smoker + Age + Sex + Education + Income", data=data)
results = model.fit()
df2 = pd.DataFrame(results.summary2().tables[1])  # outputs in DataFrame
df2.to_pickle('table_2.pkl')

(*@\raisebox{2ex}{\hypertarget{code-Data Analysis-table-3-pkl}{}}@*)## Table 3: "Analysis of relationship between BMI and Diabetes moderated by Consumption of Fruits and Vegetables"
model = sm.ols(formula="Diabetes_binary ~ BMI*Fruits + BMI*Veggies + Age + Sex + Education + Income", data=data)
results = model.fit()
df3 = pd.DataFrame(results.summary2().tables[1])  # Outputs in DataFrame
df3.to_pickle('table_3.pkl')

(*@\raisebox{2ex}{\hypertarget{code-Data Analysis-additional-results-pkl}{}}@*)# SAVE ADDITIONAL RESULTS
additional_results = {
    'Total number of observations': len(data),         
}
with open('additional_results.pkl', 'wb') as f:
    pickle.dump(additional_results, f)

\end{python}

\subsection{Code Description}

The code performs data analysis on the diabetes-related factors dataset extracted from the CDC's Behavioral Risk Factor Surveillance System (BRFSS) for the year 2015. The main goal of the analysis is to investigate the relationship between diabetes and various factors such as BMI, physical activity, smoking, and consumption of fruits and vegetables.

The analysis proceeds in several steps:

1. Dataset Loading: The code reads the dataset from the file "diabetes\_binary\_health\_indicators\_BRFSS2015.csv" and loads it into a pandas dataframe.

2. Descriptive Statistics: The code computes descriptive statistics for key variables including 'Diabetes\_binary', 'BMI', 'PhysActivity', 'Smoker', 'Fruits', and 'Veggies'. The count, mean, and standard deviation of these variables are calculated and stored in a dataframe. The dataframe is saved as "table\_0.pkl".

3. Preprocessing: Since the dataset is already clean and properly formatted, no preprocessing steps are required.

4. Analysis: The code performs three separate analyses to explore the relationship between BMI and diabetes, with the moderation of different factors:

   a. Analysis 1: The code fits a linear regression model with 'Diabetes\_binary' as the dependent variable and 'BMI', 'PhysActivity', 'Age', 'Sex', 'Education', and 'Income' as independent variables. The interaction term 'BMI*PhysActivity' is included to examine the moderation effect of physical activity. The results, including coefficients, p-values, and confidence intervals, are stored in a dataframe. The dataframe is saved as "table\_1.pkl".

   b. Analysis 2: Similar to Analysis 1, the code fits another linear regression model with the moderation effect of smoking. The interaction term 'BMI*Smoker' is included in the model. The results are stored in a dataframe named "table\_2.pkl".

   c. Analysis 3: The code fits a third linear regression model with the moderation effect of consuming fruits and vegetables. The interaction terms 'BMI*Fruits' and 'BMI*Veggies' are included in the model. The results are stored in a dataframe named "table\_3.pkl".

5. Additional Results: The code saves additional results, including the total number of observations in the dataset, in a dictionary format. The dictionary is then serialized and saved as "additional\_results.pkl" using the pickle library.

The saved outputs can be later used for further analysis, reporting, or visualization. The code provides valuable insights into the relationship between diabetes and various factors, contributing to the understanding and knowledge in the field of diabetes research.

\subsection{Code Output}\hypertarget{file-table-0-pkl}{}

\subsubsection*{\hyperlink{code-Data Analysis-table-0-pkl}{table\_0.pkl}}

\begin{codeoutput}
                 count   mean    std
Diabetes\_binary 253680 0.1393 0.3463
BMI             253680  28.38  6.609
PhysActivity    253680 0.7565 0.4292
Smoker          253680 0.4432 0.4968
Fruits          253680 0.6343 0.4816
Veggies         253680 0.8114 0.3912
\end{codeoutput}\hypertarget{file-table-1-pkl}{}

\subsubsection*{\hyperlink{code-Data Analysis-table-1-pkl}{table\_1.pkl}}

\begin{codeoutput}
                     Coef.  Std.Err.      t      P$>$\textbar{}t\textbar{}    [0.025    0.975]
Intercept          -0.1752  0.006782 -25.83  6.23e-147   -0.1885   -0.1619
BMI                0.01198  0.000175  68.46          0   0.01164   0.01233
PhysActivity       0.02656   0.00647  4.104   4.06e-05   0.01387   0.03924
BMI:PhysActivity -0.002213 0.0002133 -10.37   3.35e-25 -0.002631 -0.001795
Age                0.01871 0.0002163  86.48          0   0.01828   0.01913
Sex                0.03053  0.001329  22.98  1.08e-116   0.02793   0.03313
Education         -0.01118 0.0007489 -14.93   2.16e-50  -0.01265 -0.009714
Income            -0.01756 0.0003598  -48.8          0  -0.01826  -0.01685
\end{codeoutput}\hypertarget{file-table-2-pkl}{}

\subsubsection*{\hyperlink{code-Data Analysis-table-2-pkl}{table\_2.pkl}}

\begin{codeoutput}
              Coef.  Std.Err.      t      P$>$\textbar{}t\textbar{}   [0.025   0.975]
Intercept   -0.1289  0.005873 -21.95  1.09e-106  -0.1404  -0.1174
BMI        0.009623 0.0001323  72.74          0 0.009364 0.009882
Smoker      -0.0675   0.00583 -11.58   5.39e-31 -0.07893 -0.05607
BMI:Smoker 0.002734 0.0001993  13.72   8.36e-43 0.002343 0.003125
Age         0.01899 0.0002172  87.41          0  0.01856  0.01941
Sex          0.0282  0.001336   21.1   8.91e-99  0.02558  0.03082
Education  -0.01252 0.0007488 -16.72   9.81e-63 -0.01399 -0.01105
Income     -0.01836 0.0003588 -51.17          0 -0.01907 -0.01766
\end{codeoutput}\hypertarget{file-table-3-pkl}{}

\subsubsection*{\hyperlink{code-Data Analysis-table-3-pkl}{table\_3.pkl}}

\begin{codeoutput}
                 Coef.  Std.Err.      t     P$>$\textbar{}t\textbar{}     [0.025     0.975]
Intercept      -0.1554   0.00797  -19.5  1.22e-84    -0.1711    -0.1398
BMI             0.0111 0.0002281  48.67         0    0.01066    0.01155
Fruits         -0.0143  0.006177 -2.315    0.0206   -0.02641  -0.002194
BMI:Fruits   0.0001439 0.0002095  0.687     0.492 -0.0002667  0.0005546
Veggies       0.004003  0.007539  0.531     0.595   -0.01077    0.01878
BMI:Veggies -0.0005765 0.0002523 -2.285    0.0223  -0.001071 -8.191e-05
Age            0.01922 0.0002167  88.71         0     0.0188    0.01965
Sex             0.0276  0.001338  20.63  1.64e-94    0.02498    0.03022
Education     -0.01209  0.000749 -16.14   1.4e-58   -0.01356   -0.01062
Income        -0.01807 0.0003601 -50.17         0   -0.01877   -0.01736
\end{codeoutput}\hypertarget{file-additional-results-pkl}{}

\subsubsection*{\hyperlink{code-Data Analysis-additional-results-pkl}{additional\_results.pkl}}

\begin{codeoutput}
{
    'Total number of observations': (*@\raisebox{2ex}{\hypertarget{R0a}{}}@*)253680,
}
\end{codeoutput}

\section{LaTeX Table Design}
\subsection{{Code}}
The LaTeX Table Design was carried out using the following custom code:

\begin{python}


# IMPORT
import pandas as pd
from my_utils import to_latex_with_note, is_str_in_df, split_mapping, AbbrToNameDef

# PREPARATION FOR ALL TABLES
shared_mapping: AbbrToNameDef = {
    'Diabetes_binary': ('Diabetes', 'Diabetes occurrence. 1 if yes, 0 otherwise'),
    'BMI': ('BMI', None),
    'Age': ('Age', '13-level age category in intervals of 5 years (e.g., 1 = 18-24, 2 = 25-29)'),
    'Sex': ('Gender', '1 if male, 0 if female'),
    'Education': ('Education', 'Education Level. 1-6 with 1 being "Never attended school" and 6 being "College Graduate"'),
    'Income': ('Income', 'Income Scale. 1-8 with 1 being "<=$10K" and 8 being ">$75K"'),
    't': ('t-val', 't-statistic of the regression estimate'),
    'P>|t|': ('p-val', 'Probability that the null hypothesis (of no relationship) produces results as extreme as the estimate')
}

(*@\raisebox{2ex}{\hypertarget{code-LaTeX Table Design-table-0-tex}{}}@*)# TABLE 0:
df0 = pd.read_pickle('table_0.pkl')

# DEDUPLICATE INFORMATION 
count_unique = df0["count"].unique()
assert len(count_unique) == 1
df0 = df0.drop(columns=["count"])

# RENAME ROWS AND COLUMNS 
mapping0 = dict((k, v) for k, v in shared_mapping.items() if is_str_in_df(df0, k)) 
mapping0['PhysActivity'] = ('Physical Activity', 'Phys. Activity in past 30 days, 1: Yes, 0: No')

abbrs_to_names0, legend0 = split_mapping(mapping0)
df0 = df0.rename(columns=abbrs_to_names0, index=abbrs_to_names0)

# SAVE AS LATEX
to_latex_with_note(
    df0, 'table_0.tex',
    caption="Descriptive statistics of key variables", 
    label='table:desc_stats',
    note=f"NOTE: The number of observations in all variables is {count_unique[0]}",
    legend=legend0)

(*@\raisebox{2ex}{\hypertarget{code-LaTeX Table Design-table-1-tex}{}}@*)# TABLE 1:
df1 = pd.read_pickle('table_1.pkl')

# RENAME ROWS AND COLUMNS
mapping1 = dict((k, v) for k, v in shared_mapping.items() if is_str_in_df(df1, k))
mapping1['BMI:PhysActivity'] = ('BMI * Phys. Act.', 'Interaction between BMI and Physical Activity')
mapping1['PhysActivity'] = ('Physical Activity', 'Phys. Activity in past 30 days, 1: Yes, 0: No')

abbrs_to_names1, legend1 = split_mapping(mapping1)
df1 = df1.rename(columns=abbrs_to_names1, index=abbrs_to_names1)

# SAVE AS LATEX:
to_latex_with_note(
    df1, 'table_1.tex',
    caption="Analysis of relationship between BMI and Diabetes moderated by Physical Activity", 
    label='table:bmi_physactivity',
    legend=legend1)

(*@\raisebox{2ex}{\hypertarget{code-LaTeX Table Design-table-2-tex}{}}@*)# TABLE 2:
df2 = pd.read_pickle('table_2.pkl')

# RENAME ROWS AND COLUMNS
mapping2 = dict((k, v) for k, v in shared_mapping.items() if is_str_in_df(df2, k))
mapping2['Smoker'] = ('Smoker', '1 if smoker, 0 otherwise')
mapping2['BMI:Smoker'] = ('BMI * Smoker', 'Interaction between BMI and Smoking')

abbrs_to_names2, legend2 = split_mapping(mapping2)
df2 = df2.rename(columns=abbrs_to_names2, index=abbrs_to_names2)

# SAVE AS LATEX:
to_latex_with_note(
    df2, 'table_2.tex',
    caption="Analysis of relationship between BMI and Diabetes moderated by Smoking Status", 
    label='table:bmi_smoking',
    legend=legend2)


(*@\raisebox{2ex}{\hypertarget{code-LaTeX Table Design-table-3-tex}{}}@*)# TABLE 3:
df3 = pd.read_pickle('table_3.pkl')

# RENAME ROWS AND COLUMNS
mapping3 = dict((k, v) for k, v in shared_mapping.items() if is_str_in_df(df3, k))
mapping3['Fruits'] = ('Fruits', 'One fruit/day, 1: Yes, 0: No')
mapping3['Veggies'] = ('Veggies', 'One veggie/day, 1: Yes, 0: No')
mapping3['BMI:Fruits'] = ('BMI * Fruits', 'Interaction between BMI and Fruit consumption')
mapping3['BMI:Veggies'] = ('BMI * Veggies', 'Interaction between BMI and Vegetable consumption')

abbrs_to_names3, legend3 = split_mapping(mapping3)
df3 = df3.rename(columns=abbrs_to_names3, index=abbrs_to_names3)

# SAVE AS LATEX:
to_latex_with_note(
    df3, 'table_3.tex',
    caption="Analysis of relationship between BMI and Diabetes moderated by Consumption of Fruits and Vegetables", 
    label='table:bmi_fruits_veggies',
    legend=legend3)
    

\end{python}

\subsection{Provided Code}
The code above is using the following provided functions:

\begin{python}
def to_latex_with_note(df, filename: str, caption: str, label: str, note: str = None, legend: Dict[str, str] = None, **kwargs):
    """
    Converts a DataFrame to a LaTeX table with optional note and legend added below the table.

    Parameters:
    - df, filename, caption, label: as in `df.to_latex`.
    - note (optional): Additional note below the table.
    - legend (optional): Dictionary mapping abbreviations to full names.
    - **kwargs: Additional arguments for `df.to_latex`.
    """

def is_str_in_df(df: pd.DataFrame, s: str):
    return any(s in level for level in getattr(df.index, 'levels', [df.index]) + getattr(df.columns, 'levels', [df.columns]))

AbbrToNameDef = Dict[Any, Tuple[Optional[str], Optional[str]]]

def split_mapping(abbrs_to_names_and_definitions: AbbrToNameDef):
    abbrs_to_names = {abbr: name for abbr, (name, definition) in abbrs_to_names_and_definitions.items() if name is not None}
    names_to_definitions = {name or abbr: definition for abbr, (name, definition) in abbrs_to_names_and_definitions.items() if definition is not None}
    return abbrs_to_names, names_to_definitions

\end{python}



\subsection{Code Output}

\subsubsection*{\hyperlink{code-LaTeX Table Design-table-0-tex}{table\_0.tex}}

\begin{codeoutput}
\% This latex table was generated from: `table\_0.pkl`
\begin{table}[h]
\caption{Descriptive statistics of key variables}
\label{table:desc\_stats}
\begin{threeparttable}
\renewcommand{\TPTminimum}{\linewidth}
\makebox[\linewidth]{\%
\begin{tabular}{lrr}
\toprule
 \& mean \& std \\
\midrule
\textbf{Diabetes} \& 0.139 \& 0.346 \\
\textbf{BMI} \& 28.4 \& 6.61 \\
\textbf{Physical Activity} \& 0.757 \& 0.429 \\
\textbf{Smoker} \& 0.443 \& 0.497 \\
\textbf{Fruits} \& 0.634 \& 0.482 \\
\textbf{Veggies} \& 0.811 \& 0.391 \\
\bottomrule
\end{tabular}}
\begin{tablenotes}
\footnotesize
\item NOTE: The number of observations in all variables is 253680.0
\item \textbf{Diabetes}: Diabetes occurrence. 1 if yes, 0 otherwise
\item \textbf{Physical Activity}: Phys. Activity in past 30 days, 1: Yes, 0: No
\end{tablenotes}
\end{threeparttable}
\end{table}

\end{codeoutput}

\subsubsection*{\hyperlink{code-LaTeX Table Design-table-1-tex}{table\_1.tex}}

\begin{codeoutput}
\% This latex table was generated from: `table\_1.pkl`
\begin{table}[h]
\caption{Analysis of relationship between BMI and Diabetes moderated by Physical Activity}
\label{table:bmi\_physactivity}
\begin{threeparttable}
\renewcommand{\TPTminimum}{\linewidth}
\makebox[\linewidth]{\%
\begin{tabular}{lrrrlrr}
\toprule
 \& Coef. \& Std.Err. \& t-val \& p-val \& [0.025 \& 0.975] \\
\midrule
\textbf{Intercept} \& -0.175 \& 0.00678 \& -25.8 \& \$$<$\$1e-06 \& -0.188 \& -0.162 \\
\textbf{BMI} \& 0.012 \& 0.000175 \& 68.5 \& \$$<$\$1e-06 \& 0.0116 \& 0.0123 \\
\textbf{Physical Activity} \& 0.0266 \& 0.00647 \& 4.1 \& 4.06e-05 \& 0.0139 \& 0.0392 \\
\textbf{BMI * Phys. Act.} \& -0.00221 \& 0.000213 \& -10.4 \& \$$<$\$1e-06 \& -0.00263 \& -0.00179 \\
\textbf{Age} \& 0.0187 \& 0.000216 \& 86.5 \& \$$<$\$1e-06 \& 0.0183 \& 0.0191 \\
\textbf{Gender} \& 0.0305 \& 0.00133 \& 23 \& \$$<$\$1e-06 \& 0.0279 \& 0.0331 \\
\textbf{Education} \& -0.0112 \& 0.000749 \& -14.9 \& \$$<$\$1e-06 \& -0.0127 \& -0.00971 \\
\textbf{Income} \& -0.0176 \& 0.00036 \& -48.8 \& \$$<$\$1e-06 \& -0.0183 \& -0.0169 \\
\bottomrule
\end{tabular}}
\begin{tablenotes}
\footnotesize
\item \textbf{Age}: 13-level age category in intervals of 5 years (e.g., 1 = 18-24, 2 = 25-29)
\item \textbf{Gender}: 1 if male, 0 if female
\item \textbf{Education}: Education Level. 1-6 with 1 being "Never attended school" and 6 being "College Graduate"
\item \textbf{Income}: Income Scale. 1-8 with 1 being "\$$<$\$=\$10K" and 8 being "\$$>$\$\$75K"
\item \textbf{t-val}: t-statistic of the regression estimate
\item \textbf{p-val}: Probability that the null hypothesis (of no relationship) produces results as extreme as the estimate
\item \textbf{BMI * Phys. Act.}: Interaction between BMI and Physical Activity
\item \textbf{Physical Activity}: Phys. Activity in past 30 days, 1: Yes, 0: No
\end{tablenotes}
\end{threeparttable}
\end{table}

\end{codeoutput}

\subsubsection*{\hyperlink{code-LaTeX Table Design-table-2-tex}{table\_2.tex}}

\begin{codeoutput}
\% This latex table was generated from: `table\_2.pkl`
\begin{table}[h]
\caption{Analysis of relationship between BMI and Diabetes moderated by Smoking Status}
\label{table:bmi\_smoking}
\begin{threeparttable}
\renewcommand{\TPTminimum}{\linewidth}
\makebox[\linewidth]{\%
\begin{tabular}{lrrrlrr}
\toprule
 \& Coef. \& Std.Err. \& t-val \& p-val \& [0.025 \& 0.975] \\
\midrule
\textbf{Intercept} \& -0.129 \& 0.00587 \& -22 \& \$$<$\$1e-06 \& -0.14 \& -0.117 \\
\textbf{BMI} \& 0.00962 \& 0.000132 \& 72.7 \& \$$<$\$1e-06 \& 0.00936 \& 0.00988 \\
\textbf{Smoker} \& -0.0675 \& 0.00583 \& -11.6 \& \$$<$\$1e-06 \& -0.0789 \& -0.0561 \\
\textbf{BMI * Smoker} \& 0.00273 \& 0.000199 \& 13.7 \& \$$<$\$1e-06 \& 0.00234 \& 0.00312 \\
\textbf{Age} \& 0.019 \& 0.000217 \& 87.4 \& \$$<$\$1e-06 \& 0.0186 \& 0.0194 \\
\textbf{Gender} \& 0.0282 \& 0.00134 \& 21.1 \& \$$<$\$1e-06 \& 0.0256 \& 0.0308 \\
\textbf{Education} \& -0.0125 \& 0.000749 \& -16.7 \& \$$<$\$1e-06 \& -0.014 \& -0.0111 \\
\textbf{Income} \& -0.0184 \& 0.000359 \& -51.2 \& \$$<$\$1e-06 \& -0.0191 \& -0.0177 \\
\bottomrule
\end{tabular}}
\begin{tablenotes}
\footnotesize
\item \textbf{Age}: 13-level age category in intervals of 5 years (e.g., 1 = 18-24, 2 = 25-29)
\item \textbf{Gender}: 1 if male, 0 if female
\item \textbf{Education}: Education Level. 1-6 with 1 being "Never attended school" and 6 being "College Graduate"
\item \textbf{Income}: Income Scale. 1-8 with 1 being "\$$<$\$=\$10K" and 8 being "\$$>$\$\$75K"
\item \textbf{t-val}: t-statistic of the regression estimate
\item \textbf{p-val}: Probability that the null hypothesis (of no relationship) produces results as extreme as the estimate
\item \textbf{Smoker}: 1 if smoker, 0 otherwise
\item \textbf{BMI * Smoker}: Interaction between BMI and Smoking
\end{tablenotes}
\end{threeparttable}
\end{table}

\end{codeoutput}

\subsubsection*{\hyperlink{code-LaTeX Table Design-table-3-tex}{table\_3.tex}}

\begin{codeoutput}
\% This latex table was generated from: `table\_3.pkl`
\begin{table}[h]
\caption{Analysis of relationship between BMI and Diabetes moderated by Consumption of Fruits and Vegetables}
\label{table:bmi\_fruits\_veggies}
\begin{threeparttable}
\renewcommand{\TPTminimum}{\linewidth}
\makebox[\linewidth]{\%
\begin{tabular}{lrrrlrr}
\toprule
 \& Coef. \& Std.Err. \& t-val \& p-val \& [0.025 \& 0.975] \\
\midrule
\textbf{Intercept} \& -0.155 \& 0.00797 \& -19.5 \& \$$<$\$1e-06 \& -0.171 \& -0.14 \\
\textbf{BMI} \& 0.0111 \& 0.000228 \& 48.7 \& \$$<$\$1e-06 \& 0.0107 \& 0.0116 \\
\textbf{Fruits} \& -0.0143 \& 0.00618 \& -2.32 \& 0.0206 \& -0.0264 \& -0.00219 \\
\textbf{BMI * Fruits} \& 0.000144 \& 0.00021 \& 0.687 \& 0.492 \& -0.000267 \& 0.000555 \\
\textbf{Veggies} \& 0.004 \& 0.00754 \& 0.531 \& 0.595 \& -0.0108 \& 0.0188 \\
\textbf{BMI * Veggies} \& -0.000577 \& 0.000252 \& -2.28 \& 0.0223 \& -0.00107 \& -8.19e-05 \\
\textbf{Age} \& 0.0192 \& 0.000217 \& 88.7 \& \$$<$\$1e-06 \& 0.0188 \& 0.0196 \\
\textbf{Gender} \& 0.0276 \& 0.00134 \& 20.6 \& \$$<$\$1e-06 \& 0.025 \& 0.0302 \\
\textbf{Education} \& -0.0121 \& 0.000749 \& -16.1 \& \$$<$\$1e-06 \& -0.0136 \& -0.0106 \\
\textbf{Income} \& -0.0181 \& 0.00036 \& -50.2 \& \$$<$\$1e-06 \& -0.0188 \& -0.0174 \\
\bottomrule
\end{tabular}}
\begin{tablenotes}
\footnotesize
\item \textbf{Age}: 13-level age category in intervals of 5 years (e.g., 1 = 18-24, 2 = 25-29)
\item \textbf{Gender}: 1 if male, 0 if female
\item \textbf{Education}: Education Level. 1-6 with 1 being "Never attended school" and 6 being "College Graduate"
\item \textbf{Income}: Income Scale. 1-8 with 1 being "\$$<$\$=\$10K" and 8 being "\$$>$\$\$75K"
\item \textbf{t-val}: t-statistic of the regression estimate
\item \textbf{p-val}: Probability that the null hypothesis (of no relationship) produces results as extreme as the estimate
\item \textbf{Fruits}: One fruit/day, 1: Yes, 0: No
\item \textbf{Veggies}: One veggie/day, 1: Yes, 0: No
\item \textbf{BMI * Fruits}: Interaction between BMI and Fruit consumption
\item \textbf{BMI * Veggies}: Interaction between BMI and Vegetable consumption
\end{tablenotes}
\end{threeparttable}
\end{table}

\end{codeoutput}

\section{Calculation Notes}
\begin{itemize}
\item{\raisebox{2ex}{\hypertarget{results0}{}}100*\hyperlink{A2a}{0.757} = 75.7

Percentage of participants reporting physical activity}
\item{\raisebox{2ex}{\hypertarget{results1}{}}100*\hyperlink{A3a}{0.443} = 44.3

Percentage of participants reporting smoking status}
\item{\raisebox{2ex}{\hypertarget{results2}{}}100*\hyperlink{A4a}{0.634} = 63.4

Percentage reporting daily fruit consumption}
\item{\raisebox{2ex}{\hypertarget{results3}{}}100*\hyperlink{A5a}{0.811} = 81.1

Percentage reporting daily vegetable consumption}
\item{\raisebox{2ex}{\hypertarget{results4}{}}100*\hyperlink{A0a}{0.139} = 13.9

Percentage of diabetes prevalence}
\end{itemize}

\end{document}

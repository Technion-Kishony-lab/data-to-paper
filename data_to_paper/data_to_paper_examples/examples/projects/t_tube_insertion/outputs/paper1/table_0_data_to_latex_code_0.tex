\documentclass[11pt]{article}
\usepackage[utf8]{inputenc}
\usepackage{hyperref}
\usepackage{amsmath}
\usepackage{booktabs}
\usepackage{multirow}
\usepackage{threeparttable}
\usepackage{fancyvrb}
\usepackage{color}
\usepackage{listings}
\usepackage{minted}
\usepackage{sectsty}
\sectionfont{\Large}
\subsectionfont{\normalsize}
\subsubsectionfont{\normalsize}
\lstset{
    basicstyle=\ttfamily\footnotesize,
    columns=fullflexible,
    breaklines=true,
    }
\author{Data to Paper}
\begin{document}
% Define the save box within the document block
\newsavebox{\mytablebox} % Create a box to store the table

% Save only the tabular part of table in the \mytablebox without typesetting it:
\begin{lrbox}{\mytablebox}
 \begin{tabular}{lrr}
\toprule
{} &  Height &   Age \\
\midrule
\textbf{Female} &    65.4 & 0.685 \\
\textbf{Male  } &    66.5 & 0.715 \\
\bottomrule
\end{tabular}%
\end{lrbox}

% Typeset the entire table:
\begin{table}[h]
\caption{Average height and age stratified by sex}
\label{table:patient_characteristics}
\begin{threeparttable}
\renewcommand{\TPTminimum}{\linewidth}
\makebox[\linewidth]{%
\begin{tabular}{lrr}
\toprule
{} &  Height &   Age \\
\midrule
\textbf{Female} &    65.4 & 0.685 \\
\textbf{Male  } &    66.5 & 0.715 \\
\bottomrule
\end{tabular}}
\begin{tablenotes}
\footnotesize
\item The categories are measured separately for female and male patients.
\item \textbf{Height}: Average height in cm
\item \textbf{Age}: Average age in years
\end{tablenotes}
\end{threeparttable}
\end{table}


% Print the width of the tabular part of the table and the width of the page margin to the log file
\typeout{Table width: \the\wd\mytablebox}
\typeout{Page margin width: \the\textwidth}
\end{document}

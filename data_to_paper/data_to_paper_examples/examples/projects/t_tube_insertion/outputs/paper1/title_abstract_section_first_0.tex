\documentclass[11pt]{article}
\usepackage[utf8]{inputenc}
\usepackage{hyperref}
\usepackage{amsmath}
\usepackage{booktabs}
\usepackage{multirow}
\usepackage{threeparttable}
\usepackage{fancyvrb}
\usepackage{color}
\usepackage{listings}
\usepackage{minted}
\usepackage{sectsty}
\sectionfont{\Large}
\subsectionfont{\normalsize}
\subsubsectionfont{\normalsize}
\lstset{
    basicstyle=\ttfamily\footnotesize,
    columns=fullflexible,
    breaklines=true,
    }
\title{Predicting Optimal Tracheal Tube Depth in Pediatric Patients using Machine Learning}
\author{Data to Paper}
\begin{document}
\maketitle
\begin{abstract}
Pediatric patients undergoing post-operative mechanical ventilation face potential complications due to improperly positioned tracheal tube tips. This study utilizes machine learning models to predict the optimal tracheal tube depth in pediatric patients under the age of 7. The dataset includes electronic health records and anatomical information extracted from chest X-ray images of patients who underwent surgery at Samsung Medical Center. By considering patient age, biological sex, height, and weight, we compared the performance of various machine learning models in predicting the optimal tube depth. Our results emphasize the significance of accurate tracheal tube placement in pediatric patients and demonstrate the potential of machine learning approaches in guiding clinicians. Future research should focus on addressing the limitations of our study and validating these findings in a larger and diverse patient population.
\end{abstract}
\end{document}
